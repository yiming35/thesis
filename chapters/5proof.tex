\section{Proof of main theorem}
Here we present the proof of Theorem \ref{thm:looijenga}. Throughout this section let $X$ be a path connected topological space with basepoints $a,b$. Our strategy will be use the cosimplicial path space model to construct the $E^0$-page of a spectral sequence which (1) converges to the group $H_n(X^n, X(n)^a_b)$ and (2) is related to the cobar construction, hence the groups $\Z\pi_1(X,a)/\Z\pi_1(X,a)\mathcal{I}^{n+1}_a$.

\subsection{Building the spectral sequence} Recall the cosimplicial path space $P_{a,b}^\bullet X$ of Example \ref{example:cosimplicial_path_space}. From now we denote it by $P^\bullet$ for simplicity. Notice first that the union of the images of the coface maps $d^i: X^{n-1}\to X^n$ is exactly the subspace $X(n)^a_b$. This is a good sign.

Now consider the bicomplex $\mathscr{C}^\bullet_\bullet$ which is obtained by applying singular chains to $P^\bullet$: 
\[\mathscr{C}^q_p = C_p(P^q)\]
with ``horizontal'' differential $d^\mathscr{C}_H: C_p(P^q)\to C_{p-1}(P^q)$ given by the singular boundary map, and ``vertical'' differential $d^V_\mathscr{C}: C_p(P^q) \to C_p(P^q)$ given by the differential of the Moore complex of the cosimplicial abelian group $C_p(P^\bullet)$. 
\begin{prop}
 $C^\bullet_\bullet$ is a bicomplex.
 \begin{proof}
 That $d^\mathscr{C}_H$ and $d^V_\mathscr{C}$ both square to zero follows from the fact that each row is a chain complex and each row is a cochain complex. It remains to check commutativity of the squares: 
 
\[\begin{tikzcd}[cramped]
    % https://q.uiver.app/#q=WzAsNCxbMSwxLCJDX3AoWF5xKSJdLFsxLDAsIkNfcChYXntxKzF9KSJdLFswLDAsIkNfe3AtMX0oWF57cSsxfSkiXSxbMCwxLCJDX3twLTF9KFhee3F9KSJdLFswLDFdLFswLDNdLFszLDJdLFsxLDJdXQ==
	{C_{p-1}(X^{q+1})} & {C_p(X^{q+1})} \\
	{C_{p-1}(X^{q})} & {C_p(X^q)}
	\arrow[from=1-2, to=1-1]
	\arrow[from=2-1, to=1-1]
	\arrow[from=2-2, to=1-2]
	\arrow[from=2-2, to=2-1]
\end{tikzcd}\]
Starting with a map $f: \Delta^p \to X^q$, if we first go up then left, we get the map 
\[ \Delta^{p-1}\xrightarrow{([p]\to [p-1])^*}\Delta^p \xrightarrow{f} X^q \xrightarrow{\sum (-1)^id^i} X^{q+1}\]
and proceeding in the other direction gives the exact same map.
(This is really obvious, might not need to include).
 \end{proof}
\end{prop}

Each column $p$ of this bicomplex forms a cochain complex $\mathscr{C}_p^\bullet$ which is the Moore complex of $C_p(F^\bullet)$. Consider now the bicomplex $\mathscr{N}^\bullet_\bullet$ whose columns are instead the normalized cochain complexes of $C_p(F^\bullet)$. This is a bicomplex for the same reason as above, and the inclusion $\mathscr{N} \hookrightarrow \mathscr{C}$ is a quasi-isomorphism of bicomplexes by Dold--Kan. Let us spell out what the groups in $\mathscr{N}$ actually are. Following the definition, we have 
\[\mathscr{N}^q_p = \coker\bigoplus_{i=1}^{n} ((d^i)_*: C_p(X^{n-1})\to C^p(X^n)) = C^p(X^n) / \lb \sum_{i=1}^n \im (d^i)_*\rb  = C_p(X^n, X(n)_b).\]
Moreover, the vertical differential $\mathscr{N}^q_p \to \mathscr{N}^{q+1}_p$ is induced by the zeroth coface map $d^0$. 
% See Figure \ref{fig:bicomplex_C} for the first few rows and columns.
% \begin{figure}
%     \begin{tikzcd}[cramped]
%     % https://q.uiver.app/#q=WzAsMTUsWzAsMywiQ18wKCopIl0sWzEsMywiQ18xKCopIl0sWzIsMywiQ18yKCopIl0sWzAsMiwiQ18wKFgpIl0sWzEsMiwiQ18xKFgpIl0sWzIsMiwiQ18yKFgpIl0sWzAsMSwiQ18wKFheMikiXSxbMSwxLCJDXzAoWF4yKSJdLFsyLDEsIkNfMChYXjIpIl0sWzMsMSwiXFxjZG90cyJdLFszLDIsIlxcY2RvdHMiXSxbMywzLCJcXGNkb3RzIl0sWzIsMCwiXFx2ZG90cyJdLFsxLDAsIlxcdmRvdHMiXSxbMCwwLCJcXHZkb3RzIl0sWzMsNl0sWzAsM10sWzEsMF0sWzQsM10sWzcsNl0sWzgsN10sWzUsNF0sWzQsN10sWzIsMV0sWzIsNV0sWzEsNF0sWzUsOF0sWzksOF0sWzEwLDVdLFsxMSwyXSxbOCwxMl0sWzcsMTNdLFs2LDE0XV0=
% 	\vdots & \vdots & \vdots & \\
% 	{C_0(X^2)} & {C_0(X^2)} & {C_0(X^2)} & \cdots \\
% 	{C_0(X)} & {C_1(X)} & {C_2(X)} & \cdots \\
% 	{C_0(*)} & {C_1(*)} & {C_2(*)} & \cdots
% 	\arrow[from=2-1, to=1-1]
% 	\arrow[from=2-2, to=1-2]
% 	\arrow[from=2-2, to=2-1]
% 	\arrow[from=2-3, to=1-3]
% 	\arrow[from=2-3, to=2-2]
% 	\arrow[from=2-4, to=2-3]
% 	\arrow[from=3-1, to=2-1]
% 	\arrow[from=3-2, to=2-2]
% 	\arrow[from=3-2, to=3-1]
% 	\arrow[from=3-3, to=2-3]
% 	\arrow[from=3-3, to=3-2]
% 	\arrow[from=3-4, to=3-3]
% 	\arrow[from=4-1, to=3-1]
% 	\arrow[from=4-2, to=3-2]
% 	\arrow[from=4-2, to=4-1]
% 	\arrow[from=4-3, to=3-3]
% 	\arrow[from=4-3, to=4-2]
% 	\arrow[from=4-4, to=4-3]
% \end{tikzcd}
% \caption{The bicomplex $\mathscr{C}^\bullet_\bullet$.}
% \label{fig:bicomplex_C}
% \end{figure}

Now consider the bicomplex obtained from $\mathscr{N}$ by setting all rows above $q=n$ to zero, which we will denote by $(\mathscr{N}^{\leq n})^\bullet_\bullet$.
\begin{prop}
    Filtering $(\mathscr{N}^{\leq n})^\bullet_\bullet$ in the horizontal direction i.e. by the lower index, we obtain a spectral sequence collapsing on the second page with 
    \[E^2_{n,n} = H_n(X^n, X(n)^a_b), \quad E^2_{0,0} = \begin{cases}
        \Z & a=b\\
        0 &  a\neq b
    \end{cases}\]
    converging to $H_\bullet(\tot(\mathscr{N}^{\leq n})^\bullet_\bullet)$.
    \begin{proof}
        Convergence is given by the fact that our bicomplex is concentrated in the first quadrant. To compute the $E^2$-page, let us begin in the $E^0$-page, whose $p$th column is the following normalized cochain complex (beginning below in degree $q=0$):
        \[C_p(*)\to C_p(X, \{b\})\to \cdots\to C_p(X^n, X(n)_b)\to 0 \to \cdots\]
        of the Moore complex 
        \begin{equation}\label{eqn:moore_complex}
            C_p(*)\to C_p(X) \to \cdots \to C_p(X^n)\to 0 \to \cdots
        \end{equation}
        so they have the same cohomology. Moreover, the $E^0$-differential is the cochain differential $d^0$ given our filtration. But now (\ref{eqn:moore_complex}) is just the cochain complex obtained by taking the free abelian group on the cosimplicial set $[q] \mapsto \Map(\Delta^p, F^q)$, with terms $q>n$ set to zero. Therefore, by Proposition \ref{prop:goodwillie_cor}, we have that $E^1_{p,q} = 0$ for $q \neq 0,n$. For $q=n$ and $q=0$, we have 
        \[E^1_{p,n} = C_p(X^n, X(n)_b) /\im(d^0) = C_p(X^n, X(n)^a_b)\]
        \[E^1_{p,0} = \ker(d^0: C_p(*)\to C_p(X, \{b\})) = \begin{cases}
        \Z & p=0 , a=b \\
        0 & \text{otherwise}
        \end{cases}.\]

        On the $E^1$-page, the differential is now induced by singular boundary map. Thus on the only two possibly nonzero rows $q=n$ of the $E^1$-page, we just compute the singular homology of to obtain
        \[E^2_{p, n} = H_p(X^n, X(n)^a_b).\]
        And on row $q=0$, we have 
        \[E^2_{p, 0} = \begin{cases}
            H_p(\Z\leftarrow 0 \leftarrow \cdots) & a=b \\
            0 & a\neq b
        \end{cases}\]
        as desired. 

        At this point, because the only nonzero rows are $q=n$ and possibly $q=0$, the differentials on the pages $E^{\geq 2}$ are all zero, except possibly on page $E^{n+1}$, where we have a differential $d_{n+1}$ of bidegree $(-n-1, -n)$ from $E^{n+1}_{n+1, n} = H_{n+1}(X^n, X(n)^a_b)$ to $E^{n+1}_{0,0} = \Z$. To see that $d_{n+1} = 0$, consider the map of pairs $f: (X,x)\to (*,*)$. Repeating the constructions in this section, we get a spectral sequence $D_{\bullet,\bullet}$ computing $H_n(*^n, *(n)^*_*) = 0$. By functoriality, $f$ induces a morphism of spectral sequences. Specializing to the $n+1$th page, we have a commutative square 
		
		\[\begin{tikzcd}[cramped]
			% https://q.uiver.app/#q=WzAsOCxbMSwwLCJFXntuKzF9X3tuKzEsbn0iXSxbMSwxLCJEXntuKzF9X3tuKzEsbn0iXSxbMiwwLCJFXntuKzF9X3swLDB9Il0sWzIsMSwiRF57bisxfV97MCwwfSJdLFswLDAsIkhfe24rMX0oWF5uLFgobileeF94KSJdLFszLDAsIlxcWiJdLFswLDEsIkhfe24rMX0oKiwqKT0wIl0sWzMsMSwiXFxaIl0sWzAsMSwiZl8qIiwyXSxbMCwyLCJkXkVfe24rMX0iXSxbMSwzLCJkXkRfe24rMX0iLDJdLFsyLDMsImZfKiJdLFs0LDAsIiIsMix7InN0eWxlIjp7ImhlYWQiOnsibmFtZSI6Im5vbmUifX19XSxbNiwxLCIiLDIseyJzdHlsZSI6eyJoZWFkIjp7Im5hbWUiOiJub25lIn19fV0sWzIsNSwiIiwyLHsic3R5bGUiOnsiaGVhZCI6eyJuYW1lIjoibm9uZSJ9fX1dLFszLDcsIiIsMix7InN0eWxlIjp7ImhlYWQiOnsibmFtZSI6Im5vbmUifX19XV0=
			{H_{n+1}(X^n,X(n)^x_x)} & {E^{n+1}_{n+1,n}} & {E^{n+1}_{0,0}} & \Z \\
			{H_{n+1}(*,*)=0} & {D^{n+1}_{n+1,n}} & {D^{n+1}_{0,0}} & \Z
			\arrow[equal, from=1-1, to=1-2]
			\arrow["{d^E_{n+1}}", from=1-2, to=1-3]
			\arrow["{f_*}"', from=1-2, to=2-2]
			\arrow[equal, from=1-3, to=1-4]
			\arrow["{f_*}", from=1-3, to=2-3]
			\arrow[equal, from=2-1, to=2-2]
			\arrow["{d^D_{n+1}}"', from=2-2, to=2-3]
			\arrow[equal, from=2-3, to=2-4]
		\end{tikzcd}\]
		But now the $f_*: \Z\to\Z$ is the identity since it is the induced map $H_0(X,x)\to H_0(*,*)$, forcing $d^E_{n+1} = 0$, so $E^2 = E^{\infty}$ as desired.
    \end{proof}
\end{prop}
\begin{cor}\label{cor:spectral_convergence}
We have a surjection 
\[H_0(\tot(\mathscr{N}^{\leq n})^\bullet_\bullet) \twoheadrightarrow H_n(X^n, X(n)^a_b)\]
whose kernel is $\Z$ if $a=b$ and trivial otherwise.
\begin{proof}
    Follows from convergence of the spectral sequence.
\end{proof}
\end{cor}
\subsection{Functoriality}
It remains to compute $H_0$ of this totalization. For this we return to our original bicomplex $\mathscr{C}^\bullet_\bullet$ and instead of normalizing the columns, we first look at the rows $\mathscr{C}^q_\bullet$ for each $q$, which are the singular chain complexes $C_\bullet(X^q)$. Then we can apply the topological Alexander--Whitney map on each row:
\[\mathscr{C}^q_\bullet = C_\bullet(X^q) \to C_\bullet(X)^{\tensor q}\]
to obtain a new bicomplex, which we will call $\mathscr{D}^\bullet_\bullet$. 

\begin{prop}
	The map $\mathscr{C} \to \mathscr{D}$ induced by Alexander--Whitney is a quasi-isomorphism of bicomplexes, with inverse quasi-isomorphism induced by the Eilenberg--Zilber map.
	\begin{proof}
		Combine Corollary \ref{cor:topological_aw}, Theorem \ref{thm:eilenberg_zilber}, and Proposition \ref{lem:quasiiso_bicomplex},
	\end{proof}
\end{prop}
By definition of tensor product of chain complexes, $\mathscr{D}$ has a bigrading given by
\begin{equation}
\mathscr{D}^q_p = (C_\bullet(X)^{\tensor q})_p = \bigoplus_{n_1 + \cdots + n_q = p} C_{n_1}(X) \tensor \cdots \tensor C_{n_q}(X).
\end{equation}

\begin{thm}\label{thm:two_sided_cobar_D}
	Let $NC_\bullet(-)$ be the normalized singular chains functor. Regard $NC_\bullet(\{a\})$ as a right $NC_\bullet(X)$-comodule via the right comodule action 
	\[\Delta^R: NC_\bullet(\{a\})\xrightarrow{\AW} NC_\bullet(\{a\})\tensor NC_\bullet(\{a\}) \xrightarrow{\id \tensor \iota_a} NC_\bullet(\{a\}) \tensor NC_\bullet(X)\]
	and likewise for $C_\bullet(\{b\})$ as a left $C_\bullet(X)$-comodule. Then there is a natural isomorphism 
	\[\Cobar(NC_\bullet(\{b\}), NC_\bullet(X), NC_\bullet(\{a\})) \to \mathscr{D}.\]
	\begin{proof}
		Denote $\Cobar(\cdots)$ by $\Cobar$. Note that $NC_\bullet(\{a\})$ and $NC_\bullet(\{b\})$ are just the chain complexes with a $\Z$ term concentrated in degree zero, i.e. the identity object in the symmetric monoidal structure on $\catname{Ch}_+(\catname{Ab})$. Hence we have natural isomorphisms 
		\[\psi_a: NC_\bullet(\{a\}) \tensor NC_\bullet(X) \to NC_\bullet(X)\]
		\[\psi_b: NC_\bullet(X) \tensor NC_\bullet(\{b\}) \to NC_\bullet(X)\]
		yielding a natural isomorphism
		\begin{equation}
			\begin{split}
			\phi: \Cobar^q_p &=  \bigoplus_{n + b_1 + \cdots + b_q + m = p} NC_n(\{a\}) \tensor NC_{b_1}(X) \tensor \cdots \tensor NC_{b_q}(X)\tensor NC_m(\{b\}) \\
			&\to \bigoplus_{b_1 + \cdots + b_q = p}  NC_{b_1}(X) \tensor \cdots \tensor NC_{b_q}(X) = \mathscr{D}^q_p.
			\end{split}
		\end{equation}
		We now check that the differentials are compatible with $\phi$. Because $NC_\bullet(\{a\})$ and $NC_\bullet(\{b\})$ have all differentials zero, the horizontal cobar differential formula \ref{eqn: twoside_cobar_differential} gives 
		\begin{equation*}
		\begin{split}
			\phi(d_H^{\Cobar}(m \tensor c_1\tensor \cdots \tensor c_q\tensor n)) &= \phi(\sum_{i=1}^q (-1)^{\sigma(c_i)} m \tensor c_1 \tensor \cdots \tensor \partial_{NC}(c_i) \tensor \cdots \tensor c_q \tensor n) \\
			&= \sum_{i=1}^q (-1)^{\sigma(c_i)}  \psi_a(m\tensor c_1) \tensor \cdots \tensor \partial_{NC}(c_i) \tensor \cdots \tensor \psi_b(c_q\tensor n)  \\
			&= d_{\mathscr{D}}(\phi(m\tensor c_1\tensor \cdots \tensor c_q \tensor n)).
		\end{split}
		\end{equation*}
		The vertical differential in both cases is given by the alternating sum of coface maps. We need to show that $\phi\circ d_{\Cobar}^i = d_{\mathscr{D}}^i \circ \phi$ for all $i$. For $i=0$, the map $d_{\mathscr{D}}^i: \mathscr{D}^q_\bullet \to \mathscr{D}^{q+1}_\bullet$ is induced by the following inclusion under the topological Alexander--Whitney map: 
		\[(x_1, ..., x_q)\to (a, x_1, ..., x_q).\]
		Naturality of Alexander--Whitney implies that the following diagram commutes:
			\[\begin{tikzcd}[sep=large]
				% https://q.uiver.app/#q=WzAsNixbMSwwLCJDX1xcYnVsbGV0KCpcXHRpbWVzIFhecSkiXSxbMiwwLCJDX1xcYnVsbGV0KFhee3ErMX0pIl0sWzEsMSwiQ19cXGJ1bGxldCgqKSBcXHRlbnNvciBDX1xcYnVsbGV0KFgpXntcXHRlbnNvciBxfSJdLFsyLDEsIkNfXFxidWxsZXQoWClee1xcdGVuc29yIHErMX0iXSxbMCwwLCJDX1xcYnVsbGV0KFhecSkiXSxbMCwxLCJDX1xcYnVsbGV0KFgpXntcXHRlbnNvciBxfSJdLFsyLDMsIkNfXFxidWxsZXQoXFxpb3RhX3gpXFx0ZW5zb3IgXFxpZF57XFx0ZW5zb3IgcX0iXSxbMCwyLCJcXEFXX3sqLCBYXnF9IiwyXSxbMCwxLCJDX1xcYnVsbGV0KFxcaW90YV94XFx0aW1lc1xcaWRfe1hecX0pIl0sWzEsMywiXFxBV197cSsxfSJdLFs0LDAsIlxcY29uZyJdLFs0LDUsIlxcQVdfcSIsMl0sWzUsMiwiXFxjb25nIl0sWzQsMSwiQ19cXGJ1bGxldChkXjApIiwwLHsiY3VydmUiOi01fV0sWzUsMywiXFxtYXRoc2Nye0R9KGReMCkiLDIseyJjdXJ2ZSI6NX1dXQ==
				{NC_\bullet(X^q)} & {NC_\bullet(\{a\}\times X^q)} & {NC_\bullet(X^{q+1})} \\
				{NC_\bullet(X)^{\tensor q}} & {NC_\bullet(\{a\}) \tensor NC_\bullet(X)^{\tensor q}} & {NC_\bullet(X)^{\tensor q+1}}
				\arrow["\cong", from=1-1, to=1-2]
				\arrow["{NC_\bullet(d^0)}", curve={height=-30pt}, from=1-1, to=1-3]
				\arrow["{\AW_q}"', from=1-1, to=2-1]
				\arrow["{NC_\bullet(\iota_a\times\id_{X^q})}", from=1-2, to=1-3]
				\arrow["{\AW_{\{a\}, X^q}}"', from=1-2, to=2-2]
				\arrow["{\AW_{q+1}}", from=1-3, to=2-3]
				\arrow["\cong", from=2-1, to=2-2]
				\arrow["{d^0_{\mathscr{D}}}"', curve={height=30pt}, from=2-1, to=2-3]
				\arrow["{NC_\bullet(\iota_a)\tensor \id^{\tensor q}}", from=2-2, to=2-3]
			\end{tikzcd}\]
		where $\iota_a$ is the inclusion $\{a\} \hookrightarrow X$. But now we make the identifications 
		\[NC_\bullet(\{a\})\tensor NC_\bullet(X)^{\tensor q}\cong NC_\bullet(X)^{\tensor q}, \text{ and } NC_\bullet(\iota_a) = \eta_a\]
		to conclude 
		\[ (\eta_a\tensor \id^{\tensor q}) \circ \AW_q = \AW_{q+1} \circ  \hspace{2pt}NC_\bullet(d^0)\]
		which implies $d_{\mathscr{D}}^0 = \eta_a\tensor \id^{\tensor q}$. (By a slight abuse of notation we use $\eta_a$ to denote both the constant map at $\eta_a$ as well as the constant zero simplex at $a$). Then by naturality of the Alexander--Whitney map,
		\begin{equation*}
\begin{split}
    \phi(d^0_{\Cobar}(m \tensor c_1\tensor \cdots \tensor c_q\tensor n)) &= \phi(\Delta^R(m) \tensor c_1 \tensor \cdots \tensor c_q \tensor n) \\
    &= \phi\big(((\id\tensor \iota_a)\circ \AW(m)) \tensor c_1 \tensor \cdots \tensor c_q \tensor n \big) \\
    &= \phi(m \tensor \eta_a \tensor c_1 \tensor \cdots \tensor c_q \tensor n) \\
    &= \psi_a(m \tensor \eta_a) \tensor c_1 \tensor \cdots \tensor \psi_b(c_q \tensor n) \\
    &= \eta_a \tensor \psi_a(m \tensor c_1) \tensor \cdots \tensor \psi_b(c_q \tensor n) \\
    &= (\eta_a \tensor \id^{\tensor q})\big(\psi_a(m \tensor c_1) \tensor \cdots \tensor \psi_b(c_q \tensor n)\big) \\
    &= d^0_\mathscr{D}(\phi(m \tensor c_1\tensor \cdots \tensor c_q \tensor n))
\end{split}
\end{equation*}
		The proof for $d^1, ..., d^q$ are analogous. Hence $\phi$ is a bicomplex isomorphism $\Cobar \to \mathscr{D}$.
	% 	The coface maps $d^i: X^q \to X^{q+1}$ and codegeneracy maps $s^i: X^q\to X^{q-1}$ respectively induce maps $\mathscr{D}(d^i): \mathscr{D}^q_\bullet \to \mathscr{D}^{q+1}_\bullet, \quad \mathscr{D}(s^i): \mathscr{D}^q_\bullet \to \mathscr{D}^{q-1}_\bullet$ given by
	% \[\mathscr{D}(d^i) = \begin{cases}
	% 	\eta_a \tensor \id^{\tensor q} & i=0 \\
	% 	\id^{\tensor i-1}\tensor \Delta \tensor \id^{\tensor q-i} & i \in \{1, ..., q\} \\
	% 	\id^{\tensor q} \tensor \eta_b & i=q
	% \end{cases}
	% \quad \text{ and } \quad 
	% \mathscr{D}(s^i) = \id^{\tensor i}\tensor \epsilon\tensor \id^{q-i-1}\]
	% where $\eta_a$ and $\eta_b$ are the generators of $C_\bullet(\{a\}) = \Z = C_\bullet(\{b\})$.
		% Naturality of Alexander--Whitney implies that the following diagram commutes:
		% 	\[\begin{tikzcd}[sep=large]
		% 		% https://q.uiver.app/#q=WzAsNixbMSwwLCJDX1xcYnVsbGV0KCpcXHRpbWVzIFhecSkiXSxbMiwwLCJDX1xcYnVsbGV0KFhee3ErMX0pIl0sWzEsMSwiQ19cXGJ1bGxldCgqKSBcXHRlbnNvciBDX1xcYnVsbGV0KFgpXntcXHRlbnNvciBxfSJdLFsyLDEsIkNfXFxidWxsZXQoWClee1xcdGVuc29yIHErMX0iXSxbMCwwLCJDX1xcYnVsbGV0KFhecSkiXSxbMCwxLCJDX1xcYnVsbGV0KFgpXntcXHRlbnNvciBxfSJdLFsyLDMsIkNfXFxidWxsZXQoXFxpb3RhX3gpXFx0ZW5zb3IgXFxpZF57XFx0ZW5zb3IgcX0iXSxbMCwyLCJcXEFXX3sqLCBYXnF9IiwyXSxbMCwxLCJDX1xcYnVsbGV0KFxcaW90YV94XFx0aW1lc1xcaWRfe1hecX0pIl0sWzEsMywiXFxBV197cSsxfSJdLFs0LDAsIlxcY29uZyJdLFs0LDUsIlxcQVdfcSIsMl0sWzUsMiwiXFxjb25nIl0sWzQsMSwiQ19cXGJ1bGxldChkXjApIiwwLHsiY3VydmUiOi01fV0sWzUsMywiXFxtYXRoc2Nye0R9KGReMCkiLDIseyJjdXJ2ZSI6NX1dXQ==
		% 		{C_\bullet(X^q)} & {C_\bullet(*\times X^q)} & {C_\bullet(X^{q+1})} \\
		% 		{C_\bullet(X)^{\tensor q}} & {C_\bullet(*) \tensor C_\bullet(X)^{\tensor q}} & {C_\bullet(X)^{\tensor q+1}}
		% 		\arrow["\cong", from=1-1, to=1-2]
		% 		\arrow["{C_\bullet(d^0)}", curve={height=-30pt}, from=1-1, to=1-3]
		% 		\arrow["{\AW_q}"', from=1-1, to=2-1]
		% 		\arrow["{C_\bullet(\iota_a\times\id_{X^q})}", from=1-2, to=1-3]
		% 		\arrow["{\AW_{*, X^q}}"', from=1-2, to=2-2]
		% 		\arrow["{\AW_{q+1}}", from=1-3, to=2-3]
		% 		\arrow["\cong", from=2-1, to=2-2]
		% 		\arrow["{\mathscr{D}(d^0)}"', curve={height=30pt}, from=2-1, to=2-3]
		% 		\arrow["{C_\bullet(\iota_a)\tensor \id^{\tensor q}}", from=2-2, to=2-3]
		% 	\end{tikzcd}\]
		% where $\iota_a$ is the inclusion $\{a\} \hookrightarrow X$. But now we make the identifications 
		% \[C_\bullet(*)\tensor C_\bullet(X)^{\tensor q}\cong C_\bullet(X)^{\tensor q}, \text{ and } C_\bullet(\iota_a) = \eta_a\]
		% to conclude 
		% \[ (\eta(1)\tensor \id^{\tensor q}) \circ \AW_q = \AW_{q+1} \circ  \hspace{2pt}C_\bullet(d^0)\]
		% which implies $\mathscr{D}(d^0) = \eta_a\tensor \id^{\tensor q}$. The arguments for the other maps are analogous.
	\end{proof}
\end{thm}
Using this theorem and the bicomplex quasi-isomorphisms constructed previosuly, this result follows:
\begin{thm}\label{thm:new_cobar}
    There is a quasi-isomorphism of bicomplexes
	\[\Cobar(NC_\bullet(\{b\}), NC_\bullet(X), NC_\bullet(\{a\})) \to \mathscr{N}.\]
    \begin{proof}
    Composing the quasi-isomorphisms and isomorphisms obtained so far, we have 
		% https://q.uiver.app/#q=WzAsNCxbMCwwLCJcXENvYmFyKENfXFxidWxsZXQoXFx7YlxcfSksIENfXFxidWxsZXQoWCksIENfXFxidWxsZXQoXFx7YVxcfSkpIl0sWzEsMCwiXFxtYXRoc2Nye0R9Il0sWzIsMCwiXFxtYXRoc2Nye0N9Il0sWzMsMCwiXFxtYXRoc2Nye059Il0sWzAsMV0sWzEsMiwiXFxBVyIsMCx7Im9mZnNldCI6LTEsInN0eWxlIjp7InRhaWwiOnsibmFtZSI6ImFycm93aGVhZCJ9LCJoZWFkIjp7Im5hbWUiOiJub25lIn19fV0sWzIsMSwiXFxFWiIsMCx7Im9mZnNldCI6LTEsInN0eWxlIjp7InRhaWwiOnsibmFtZSI6ImFycm93aGVhZCJ9LCJoZWFkIjp7Im5hbWUiOiJub25lIn19fV0sWzIsMywiKFxcUzUuMSkiXSxbMCwzLCIiLDIseyJjdXJ2ZSI6NSwic3R5bGUiOnsiYm9keSI6eyJuYW1lIjoiZGFzaGVkIn19fV1d
		\[\begin{tikzcd}[column sep=3em]
			{\Cobar(NC_\bullet(\{b\}), NC_\bullet(X), NC_\bullet(\{a\}))} & {\mathscr{D}} & {\mathscr{C}} & {\mathscr{N}}
			\arrow["\ref{thm:two_sided_cobar_D}", from=1-1, to=1-2]
			\arrow["\AW", shift left, tail reversed, no head, from=1-2, to=1-3]
			\arrow["\EZ", shift left, tail reversed, no head, from=1-3, to=1-2]
			\arrow["{(\S5.1)}", from=1-3, to=1-4]
		\end{tikzcd}\]
		which is a quasi-isomorphism.
    \end{proof}
\end{thm}
\begin{rem}
We have not yet used Adams' theorem. But what we have proven should suggest that the theorem is true for the following reason. We have exhibited a quasi-isomorphism between the two-sided cobar construction on $C_\bullet(X)$ and the totalization of $\mathscr{C}$, which is itself the bicomplex constructed by taking singular chains on the cosimplicial path space! So if we can somehow commute ``chains on a totalization'' and ``totalization of chains'' as in the right arrow in the following diagram
\[\begin{tikzcd}[cramped]
	{\Cobar(C_\bullet(X))} && {\tot(\mathscr{C})=\tot(C_\bullet P^\bullet_{x,x}X)} \\
	{C_\bullet(\Omega_xX)} && {C_\bullet(\tot(P^\bullet_{x,x}X))}
	\arrow["\simeq", from=1-1, to=1-3]
	\arrow["{\text{Adams' theorem}}"', dashed, from=1-1, to=2-1]
	\arrow["{?}", dashed, from=1-3, to=2-3]
	\arrow[equal, from=2-1, to=2-3]
\end{tikzcd}\]
then we would have a version of Adams' theorem.
\end{rem}

% Now we have all the ingredients. It's time to cook.
\begin{proof}[Proof of Theorem \ref{thm:looijenga}]
    In the quasi-isomorphism $\Cobar\to \mathscr{N}$ produced in Theorem \ref{thm:new_cobar}, the truncation of $\mathscr{N}$ by $q$ corresponds to the tensor length truncation of $\Cobar$. It follows by Lemma \ref{lem:quasiiso_bicomplex} that the induced map on totalizations of the truncated bicomplexes 
    \[\tot\Cobar_{\leq n}\to \tot_\bullet(\mathscr{N}^{\leq n})\]
    is still a quasi-isomorphism. On zeroth homology, we therefore have an isomorphism
    \[H_0\tot\Cobar_{\leq n}\to H_0\tot_\bullet(\mathscr{N}^{\leq n})\]
	LAST STEP: 
    which by (???) yields an isomorphism 
    \[\Z\pi_X(a,b)/ \Z\pi_X(a,b)\mathcal{I}^{n+1} \to H_0\tot_\bullet(\mathscr{N}^{\leq n}).\]
    By Corollary \ref{cor:spectral_convergence}, we get a surjection
    \[\Z\pi_X(a,b)/ \Z\pi_X(a,b)\mathcal{I}^{n+1} \to H_0\tot_\bullet(\mathscr{N}^{\leq n})\twoheadrightarrow H_n(X^n, X(n)^x_x)\]
    whose kernel is $\Z$ is $a=b$ and trivial otherwise, completing the proof.
\end{proof}