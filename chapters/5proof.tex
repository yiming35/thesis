\section{The proof}
Here we present the proof of Theorem \ref{thm:looijenga}. Throughout this section let $X$ be a path connected topological space with basepoints $a,b$. Our strategy will be use the cosimplicial path space model to construct the $E^0$-page of a spectral sequence which (1) converges to the group $H_n(X^n, X(n)^a_b)$ and (2) is related to the cobar construction, hence the groups $\Z\pi_1(X,a)/\Z\pi_1(X,a)\mathcal{I}^{n+1}_a$.

\subsection{This spectral sequence...} Recall the cosimplicial path space $P_{a,b}^\bullet X$ of Example \ref{example:cosimplicial_path_space}. From now we denote it by $P^\bullet$ for simplicity. Notice first that the union of the images of the coface maps $d^i: X^{n-1}\to X^n$ is exactly the subspace $X(n)^a_b$. This is a good sign.

Now consider the bicomplex $\mathscr{C}^\bullet_\bullet$ which is morally obtained by applying singular chains to $P^\bullet$: 
\[\mathscr{C}^q_p = C_p(P^q)\]
with ``horizontal'' differential $d^\mathscr{C}_H: C_p(P^q)\to C_{p-1}(P^q)$ given by the singular boundary map, and ``vertical'' differential $d^V_\mathscr{C}: C_p(P^q) \to C_p(P^q)$ given by the differential of the Moore complex of the cosimplicial abelian group $C_p(P^\bullet)$. 
\begin{prop}
 $C^\bullet_\bullet$ is a bicomplex.
 \begin{proof}
 That $d^\mathscr{C}_H$ and $d^V_\mathscr{C}$ both square to zero follows from the fact that each row is a chain complex and each row is a cochain complex. It remains to check commutativity of the squares: 
 
\[\begin{tikzcd}[cramped]
    % https://q.uiver.app/#q=WzAsNCxbMSwxLCJDX3AoWF5xKSJdLFsxLDAsIkNfcChYXntxKzF9KSJdLFswLDAsIkNfe3AtMX0oWF57cSsxfSkiXSxbMCwxLCJDX3twLTF9KFhee3F9KSJdLFswLDFdLFswLDNdLFszLDJdLFsxLDJdXQ==
	{C_{p-1}(X^{q+1})} & {C_p(X^{q+1})} \\
	{C_{p-1}(X^{q})} & {C_p(X^q)}
	\arrow[from=1-2, to=1-1]
	\arrow[from=2-1, to=1-1]
	\arrow[from=2-2, to=1-2]
	\arrow[from=2-2, to=2-1]
\end{tikzcd}\]
Starting with a map $f: \Delta^p \to X^q$, if we first go up then left, we get the map 
\[ \Delta^{p-1}\xrightarrow{([p]\to [p-1])^*}\Delta^p \xrightarrow{f} X^q \xrightarrow{\sum (-1)^id^i} X^{q+1}\]
and proceeding in the other direction gives the exact same map.
(This is really obvious, might not need to include).
 \end{proof}
\end{prop}

Each column $p$ of this bicomplex forms a cochain complex $\mathscr{C}_p^\bullet$ which is the Moore complex of $C_p(F^\bullet)$. Consider now the bicomplex $\mathscr{N}^\bullet_\bullet$ whose columns are instead the normalized cochain complexes of $C_p(F^\bullet)$. This is a bicomplex for the same reason as above, and the inclusion $\mathscr{N} \hookrightarrow \mathscr{C}$ is a quasi-isomorphism (?) of bicomplexes by Dold--Kan. Let us spell out what the groups in $\mathscr{N}$ actually are. Following the definition, we have 
\[\mathscr{N}^q_p = \coker\bigoplus_{i=1}^{n} ((d^i)_*: C_p(X^{n-1})\to C^p(X^n)) = C^p(X^n) / \lb \sum_{i=1}^n \im (d^i)_*\rb  = C_p(X^n, X(n)_b).\]
Moreover, the vertical differential $\mathscr{N}^q_p \to \mathscr{N}^{q+1}_p$ is induced by the zeroth coface map $d^0$. 
% See Figure \ref{fig:bicomplex_C} for the first few rows and columns.
% \begin{figure}
%     \begin{tikzcd}[cramped]
%     % https://q.uiver.app/#q=WzAsMTUsWzAsMywiQ18wKCopIl0sWzEsMywiQ18xKCopIl0sWzIsMywiQ18yKCopIl0sWzAsMiwiQ18wKFgpIl0sWzEsMiwiQ18xKFgpIl0sWzIsMiwiQ18yKFgpIl0sWzAsMSwiQ18wKFheMikiXSxbMSwxLCJDXzAoWF4yKSJdLFsyLDEsIkNfMChYXjIpIl0sWzMsMSwiXFxjZG90cyJdLFszLDIsIlxcY2RvdHMiXSxbMywzLCJcXGNkb3RzIl0sWzIsMCwiXFx2ZG90cyJdLFsxLDAsIlxcdmRvdHMiXSxbMCwwLCJcXHZkb3RzIl0sWzMsNl0sWzAsM10sWzEsMF0sWzQsM10sWzcsNl0sWzgsN10sWzUsNF0sWzQsN10sWzIsMV0sWzIsNV0sWzEsNF0sWzUsOF0sWzksOF0sWzEwLDVdLFsxMSwyXSxbOCwxMl0sWzcsMTNdLFs2LDE0XV0=
% 	\vdots & \vdots & \vdots & \\
% 	{C_0(X^2)} & {C_0(X^2)} & {C_0(X^2)} & \cdots \\
% 	{C_0(X)} & {C_1(X)} & {C_2(X)} & \cdots \\
% 	{C_0(*)} & {C_1(*)} & {C_2(*)} & \cdots
% 	\arrow[from=2-1, to=1-1]
% 	\arrow[from=2-2, to=1-2]
% 	\arrow[from=2-2, to=2-1]
% 	\arrow[from=2-3, to=1-3]
% 	\arrow[from=2-3, to=2-2]
% 	\arrow[from=2-4, to=2-3]
% 	\arrow[from=3-1, to=2-1]
% 	\arrow[from=3-2, to=2-2]
% 	\arrow[from=3-2, to=3-1]
% 	\arrow[from=3-3, to=2-3]
% 	\arrow[from=3-3, to=3-2]
% 	\arrow[from=3-4, to=3-3]
% 	\arrow[from=4-1, to=3-1]
% 	\arrow[from=4-2, to=3-2]
% 	\arrow[from=4-2, to=4-1]
% 	\arrow[from=4-3, to=3-3]
% 	\arrow[from=4-3, to=4-2]
% 	\arrow[from=4-4, to=4-3]
% \end{tikzcd}
% \caption{The bicomplex $\mathscr{C}^\bullet_\bullet$.}
% \label{fig:bicomplex_C}
% \end{figure}

Now consider the bicomplex obtained from $\mathscr{N}$ by setting all rows above $q=n$ to zero, which we will denote by $(\mathscr{N}^{\leq n})^\bullet_\bullet$.
\begin{prop}
    Filtering $(\mathscr{N}^{\leq n})^\bullet_\bullet$ in the horizontal direction i.e. by the lower index, we obtain a spectral sequence collapsing on the second page with 
    \[E^2_{n,n} = H_n(X^n, X(n)^a_b), \quad E^2_{0,0} = \begin{cases}
        \Z & a=b\\
        0 &  a\neq b
    \end{cases}\]
    converging to $H_\bullet(\tot(\mathscr{N}^{\leq n})^\bullet_\bullet)$.
    \begin{proof}
        Convergence is given by the fact that our bicomplex is concentrated in the first quadrant. To compute the $E^2$-page, let us begin in the $E^0$-page, whose $p$th column is the following normalized cochain complex (beginning below in degree $q=0$):
        \[C_p(*)\to C_p(X, \{b\})\to \cdots\to C_p(X^n, X(n)_b)\to 0 \to \cdots\]
        of the Moore complex 
        \begin{equation}\label{eqn:moore_complex}
            C_p(*)\to C_p(X) \to \cdots \to C_p(X^n)\to 0 \to \cdots
        \end{equation}
        so they have the same cohomology. Moreover, the $E^0$-differential is the cochain differential $d^0$ given our filtration. But now (\ref{eqn:moore_complex}) is just the cochain complex obtained by taking the free abelian group on the cosimplicial set $[q] \mapsto \Map(\Delta^p, F^q)$, with terms $q>n$ set to zero. Therefore, by Proposition \ref{prop:goodwillie_cor}, we have that $E^1_{p,q} = 0$ for $q \neq 0,n$. For $q=n$ and $q=0$, we have 
        \[E^1_{p,n} = C_p(X^n, X(n)_b) /\im(d^0) = C_p(X^n, X(n)^a_b)\]
        \[E^1_{p,0} = \ker(d^0: C_p(*)\to C_p(X, \{b\})) = \begin{cases}
        \Z & p=0 , a=b \\
        0 & \text{otherwise}
        \end{cases}.\]

        On the $E^1$-page, the differential is now induced by singular boundary map. Thus on the only two possibly nonzero rows $q=n$ of the $E^1$-page, we just compute the singular homology of to obtain
        \[E^2_{p, n} = H_p(X^n, X(n)^a_b).\]
        And on row $q=0$, we have 
        \[E^2_{p, 0} = \begin{cases}
            H_p(\Z\leftarrow 0 \leftarrow \cdots) & a=b \\
            0 & a\neq b
        \end{cases}\]
        as desired. 

        At this point, the only possibly nonzero entries on the $E^2$-page are on row $q=n$ and $q=0$. Since the $d^2$ differential has bidegree $(-2, -1)$, the only possibly nontrivial differential occurs when $n=q=1$ and $a=b$, in which case we need to check that the map 
        \[E^2_{2,1} = H_2(X, \{a,b\}) = H_2(X, \{b\}) \to \Z = E^2_{0,0}\]
        has trivial image. 
    \end{proof}
\end{prop}
\begin{cor}
We have a surjection 
\[H_0(\tot(\mathscr{N}^{\leq n})^\bullet_\bullet) \twoheadrightarrow H_n(X^n, X(n)^a_b)\]
whose kernel is $\Z$ if $a=b$ and trivial otherwise.
\begin{proof}
    Follows from convergence of the spectral sequence.
\end{proof}
\end{cor}
\subsection{... (sort of) computes cobar}
It remains to compute $H_0$ of this totalization. For this we return to our original bicomplex $\mathscr{C}^\bullet_\bullet$ and instead of normalizing we modify it in another way using the Alexander--Whitney map.
We claim that this totalization is isomorphic to H0 of truncated cobar! 

To see this, apply AW map to C, then see that this is exactly the cobar bicomplex 

Use gadish's theorem, conclude
