\section{Adams' cobar construction}
In this chapter we define the cobar construction of a coalgebra and build up to Adams' remarkable 1956 theorem \cite{AdamsCobar}, which (roughly) says that given a based topological space $(X,x)$, one can recover the singular chains of the based loop space $\Omega_xX$ via a purely algebraic construction on the singular chains on $X$. We conclude by highlighting some recent refinements of Adams' theorem, which will be used in the proof of Looijenga's theorem. 

\subsection{The cobar construction}
Let $R$ be a commutative unital ring and let $(C = \overline{C} \oplus R,\Delta)$ be a dg coaugmented counital coassociative $R$-coalgebra. The {cobar construction} is a functor from this category of coalgebras to the category of dg associative algebras. There are many references for the following definition, including Adams' original paper. We follow the modern presentation given in \cite{Rivera2022}.
\begin{defn}
    Let $(C, \Delta, \partial)$ be as above. The \emph{cobar construction} on $C$ is the dg $R$-algebra whose underlying algebra is the tensor algebra over the desuspension of the reduced coalgebra $C$:
    \[\Cobar(C) := T(s^{-1}\overline{C}) = R \oplus s^{-1}\overline{C} \oplus s^{-1}\overline{C}^{\tensor 2}\oplus \cdots\]
    and whose differential is given by extending the linear map 
    % should this be reduced coprod or what i have now
    \[-s^{-1}\circ \partial \circ s^{+1} + (s^{-1}\tensor s^{-1})\circ \Delta \circ s^{+1}: s^{-1}\overline{C} \to s^{-1}\overline{C} \oplus s^{-1}\overline{C}^{\tensor 2}\]
    to all of $T(s^{-1}\overline{C})$, which yields a linear map of degree $-1$ from $T(s^{-1}\overline{C})$ to itself.
\end{defn}
We've not been explicit about the differential graded structure on $\Cobar(C)$, but this definition is valid owing to the following proposition:
\begin{prop}\label{prop:maps_extend}
    Let $M$ be an $R$-module. Any linear map $M \to TM$ extends uniquely to a derivation $TM\to TM$.
    \begin{proof}
        This follows from the more general Proposition 1.1.8, \cite{Loday2012}.
    \end{proof} 
\end{prop}

Regardless, it will be helpful to be a little more explicit. Notice that in the cobar differential, we utilize both the internal differential $\partial$ of $C$ as well as the coproduct $\Delta$. This suggests that upon expanding definitions, we should be able to realize the cobar construction as the totalization of a bicomplex. This is mostly a tedious exercise in indexing, but it will turn out to be useful later.

\begin{defn}
    Let $(C, \Delta, \partial)$ be as above. The \emph{cobar bicomplex} is the bicomplex $\Cobar_{\bullet}^\bullet(C)$ with 
    \[\Cobar_p^q = \bigoplus_{n_1 + \cdots + n_q = p} (s^{-1}\overline{C})_{n_1} \tensor \cdots\tensor (s^{-1}\overline{C})_{n_q}\]
    and differentials
    \[d_H: \Cobar_{p,q}\to \Cobar_{p-1}^q,\quad x_1 \tensor \cdots x_q \mapsto \sum_{i=1}^q(-1)^{\sigma(x_i)} x_1 \tensor \cdots \tensor \partial(x_i)\tensor \cdots \tensor x_q\]
    \[d_V: \Cobar_p^q \to \Cobar_p^{q+1}, \quad x_1\tensor \cdots \tensor x_q \mapsto \sum_{i=1}^q(-1)^{\sigma(x_i)} x_1 \tensor \cdots \tensor \overline{\Delta}(x_i)\tensor \cdots \tensor x_q\]
    where the $x_i$ are elements of $(s^{-1}\overline{C})_{n_i}$ and $\sigma(x_i)$ is the Koszul sign $\deg x_1 + \cdots + \deg x_{i-1}$.
\end{defn}
\begin{prop}
The totalization of $\Cobar^\bullet_\bullet(C)$ given by 
\[\tot_k(\Cobar^\bullet_\bullet) = \bigoplus_{p-q = k} \Cobar^q_p\]
with differential $\partial_{\tot} = d_H+(-1)^qd_V$ on $\Cobar^q_p$, is isomorphic as a dg algebra to $\Cobar(C_\bullet)$.
\begin{proof}
    Here the isomorphism is actually equality. We first check that the underlying $R$-algebras are the same:
    \begin{equation}
        \begin{split}
            \tot_\bullet(\Cobar^\bullet_\bullet) &= \bigoplus_{k\in \Z} \bigoplus_{p-q = k} \Cobar^q_p \\
            &= \bigoplus_{q \geq 0} \bigoplus_{p\geq 0}\Cobar^q_p \\
            &= \bigoplus_{q \geq 0}  \bigoplus_{p\geq 0}\bigoplus_{n_1 + \cdots + n_q = p} (s^{-1}\overline{C})_{n_1} \tensor \cdots\tensor (s^{-1}\overline{C})_{n_q} \\
            &= \bigoplus_{q\geq 0} (s^{-1}\overline{C})_q \\
            &= \Cobar(C_\bullet(X)).
        \end{split}
    \end{equation}
    Now by Proposition \ref{prop:maps_extend}, it suffices to check that $\partial_{\tot}$ extends the map $s^{-1}\overline{C} \to s^{-1}\overline{C} \oplus s^{-1}\overline{C}^{\tensor 2}$ given in the definition of the cobar complex. Indeed, we have that on
    \[s^{-1}\overline{C} = \bigoplus_{n_1 =p\geq 0} (s^{-1}\overline{C})_{n_1} =\bigoplus_{p\geq 0} \Cobar^1_p,\]
    the differential is $\partial_{\tot} = d_H - d_V$ which is exactly the cobar differential.
\end{proof}
\end{prop}

What is miraculous is that this purely algebraic construction encodes a good amount of topological information. This is the content of Adams' theorem:
\begin{thm}[\cite{AdamsCobar}]
    Let $(X,x)$ be a simply connected based topological space. Let $\Omega_xX$ be the space of loops in $X$ based at $x$. Then there is a natural isomorphism
    \[H_\bullet(\Cobar(C_\bullet(X)))\cong H_\bullet(\Omega_xX).\]
\end{thm}
Thus we can regard the cobar construction as the algebraic analogue of the loop space functor. Some stuff about classifying spaces, adjointness, etc. Bar construction, etc

\subsection{Refinements}
The isomorphism of Adams' theorem actually comes from a chain-level quasi-isomorphism. Moreover, the assumption that $X$ is simply connected can be removed. 
\begin{thm}[\cite{Rivera2022}, Theorem 1]
    Let $(X,x)$ be a path connected based topological space. Let $\Omega_xX$ be the space of loops in $X$ based at $x$. Then $\Cobar(C_\bullet(X))$ is quasi-isomorphic to $C_\bullet(\Omega_xX)$ as dg algebras. 
\end{thm}
\begin{rem}
    We have stated the theorem for singular chains $C_\bullet(X)$ rather than some modified chains $C_\bullet(X,b)$ (not the relative chains!), but these are quasi-isomorphic? 
\end{rem}
Moreover, there is a canonical way in which the cobar construction is related to the augmentation ideal-adic truncations of the fundamental group ring:
\begin{thm}[\cite{gadish_letterbraiding}, Theorem 1.2(3)]
    Let $A$ be a PID, and $\Gamma$ a group. Let $\mathcal{I}$ be the augemntation ideal of the group ring $A[\Gamma]$. Then we have the following isomorphism 
    \[(\mathcal{I}^n)^{\perp} \cong H^0(\Barr_{<n}(C^\bullet(B\Gamma)); A)\]
    where the left hand side is the orthogonal complement of $\mathcal{I}^n < A[\Gamma]$, and the left hand side is the zeroth cohomology of the subcoalgebra of the bar construction on the cochains of the classifying space $B\Gamma$, consisting of weight $<n$ tensors. \hfill \qed
\end{thm}
Taking $A=\Z$, $\Gamma = \pi_1(X,x)$, and dualizing, we obtain the following:
\begin{cor}
    There is an isomorphism of groups
    \[\Z\pi_1(X,x)/ \mathcal{I}^{n+1} \cong H_0(\Cobar_{\leq n}(C_\bullet(B\pi_1(X,x))))\]
    and consequently 
    \[\Z\pi_1(X,x)/ \mathcal{I}^{n+1} \cong H_0(\Cobar_{\leq n}(C_\bullet(X))).\]
    \begin{proof}
        The first statement follows immediately from Gadish's theorem. The second statement follows since the zeroth homology of the cobar construction depends only on the fundamental group of a space. To see this,
    \end{proof}
\end{cor}