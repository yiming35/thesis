\section{Introduction}

\subsection{The theorem}
Let $X$ be a path-connected topological space and choose basepoints $a, b \in X$. Let $\pi_X(a,b)$ be the set of homotopy classes of paths in $X$ from $a$ to $b$. In particular when $a=b$ we obtain the fundamental group of $X$. Notice that $\pi_1(X,a)$ acts on $\pi_X(a,b)$ on the right by concatenation of paths, so that the free abelian group $\Z\pi_X(a,b)$ is a right $\Z\pi_1(X)$-module. Let $\mathcal{I}_a\subset \Z\pi_1(X,a)$ be the augmentation ideal. 

Consider now the following subspace of $X^n$
\[X(n)^a_b := \{(x_1, ..., x_n) \in X^n: \begin{cases}
x_1 = a, \text{ or} \\
x_i = x_{i+1} \text{ for $i \in \{1, ...,n \}$, or}\\
x_n = b
\end{cases} \}.\]

\begin{thm}
There is a map 
\[\Z\pi_X(a,b)/ \Z\pi_X(a,b)\mathcal{I}^{n+1}_a\to H_n(X^n, X(n)^a_b; \Z)\]
which is a surjection with kernel $\Z$ if $a=b$ and an isomorphism if $a\neq b$.
\end{thm}
The author first encountered this theorem in a paper of Looijenga (\cite{Looijenga2025}, Theorem 1.1), who attributes this theorem (or a dual version with $\Q$-coefficients) to Deligne and Goncharov, (\cite{DELIGNE2005}, Proposition 3.4). In this paper, the authors attribute it to Beilinson.


\subsection{Outline}
