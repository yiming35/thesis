\section{Adams' cobar construction}
In this chapter we define the cobar construction of a coalgebra and introduce Adams' remarkable 1956 theorem \cite{AdamsCobar}, which (roughly) says that given a based topological space $(X,x)$, one can recover the singular chains of the based loop space $\Omega_xX$ via a purely algebraic construction on the singular chains on $X$. We then extend the cobar construction to account for two basepoints by using comodules over a coalgebra, and conclude by elaborating on the relationship between the cobar construction and the fundamental group.

\subsection{The cobar construction}
Let $R$ be a commutative unital ring and let $(C = \overline{C} \oplus R,\Delta)$ be a dg coaugmented counital coassociative $R$-coalgebra. The {cobar construction} is a functor from this category of coalgebras to the category of dg associative algebras. There are many references for the following definition, including Adams' original paper \cite{AdamsCobar} and \cite{Chen1973}. We follow the presentation given in \cite{Rivera2022}.
\begin{defn}
    Let $(C, \Delta, \partial)$ be as above. The \emph{cobar construction} on $C$ is the dg $R$-algebra whose underlying algebra is the tensor algebra over the desuspension of the reduced coalgebra $C$:
    \[\Cobar(C) := T(s^{-1}\overline{C}) = R \oplus s^{-1}\overline{C} \oplus s^{-1}\overline{C}^{\tensor 2}\oplus \cdots\]
    and whose differential is given by extending the linear map 
    % should this be reduced coprod or what i have now
    \[-s^{-1}\circ \partial \circ s^{+1} + (s^{-1}\tensor s^{-1})\circ \Delta \circ s^{+1}: s^{-1}\overline{C} \to s^{-1}\overline{C} \oplus s^{-1}\overline{C}^{\tensor 2}\]
    to all of $T(s^{-1}\overline{C})$, which yields a linear map of degree $-1$ from $T(s^{-1}\overline{C})$ to itself.
\end{defn}
The definition of the differential here is valid owing to Proposition \ref{prop:maps_extend}, which says that it suffices to define the differential on the component $s^{-1}\overline{C} \subset T(s^{-1}\overline{C})$. Regardless, it will be helpful to be a little more explicit. By the discussion in the previous chapter we have that as a chain complex, $\Cobar(C)$ has terms 
\begin{equation}
\begin{split}
    \Cobar_k(C) &= T_k(s^{-1}\overline{C}) \\
    &= \bigoplus_{q \geq 0} (s^{-1}\overline{C})^{\tensor q}_k\\
    &= \bigoplus_{q \geq 0}\bigoplus_{n_1 + \cdots + n_q = k} (s^{-1}\overline{C})_{n_1} \tensor \cdots \tensor (s^{-1}\overline{C})_{n_q} \\
    &= \bigoplus_{q \geq 0}\bigoplus_{n_1 + \cdots + n_q = k} \overline{C}_{n_1+1} \tensor \cdots \tensor \overline{C}_{n_q+1} \\
    &= \bigoplus_{q \geq 0}\bigoplus_{m_1 + \cdots + m_q = k+q} \overline{C}_{m_1} \tensor \cdots \tensor \overline{C}_{m_q} \\
    &=  \bigoplus_{p-q = k} \bigoplus_{m_1 + \cdots + m_q = p} \overline{C}_{m_1} \tensor \cdots \tensor \overline{C}_{m_q} \\
\end{split}
\end{equation}
The purpose of introducing the dummy variable $p$ at the end of this tedious indexing exercise was to suggest that we should be able to realize the cobar construction as the totalization of a bicomplex in $p$ and $q$. Indeed, in the cobar differential, we utilize both the internal differential $\partial$ of $C$ as well as the coproduct $\Delta$. One of these will turn out to raise $q$ and the other will lower $p$.

\begin{defn}\label{def:cobar_bicomplex}
    Let $(C, \Delta, \partial)$ be as above. The \emph{cobar bicomplex} is the bicomplex $\Cobar_{\bullet}^\bullet(C)$ with 
    \[\Cobar_p^q := \bigoplus_{m_1 + \cdots + m_q = p} \overline{C}_{m_1} \tensor \cdots \tensor \overline{C}_{m_q} \]
    and differentials
    \[d_H: \Cobar_{p,q}\to \Cobar_{p-1}^q,\quad x_1 \tensor \cdots x_q \mapsto \sum_{i=1}^q(-1)^{\sigma(x_i)} x_1 \tensor \cdots \tensor \partial(x_i)\tensor \cdots \tensor x_q\]
    \[d_V: \Cobar_p^q \to \Cobar_p^{q+1}, \quad x_1\tensor \cdots \tensor x_q \mapsto \sum_{i=1}^q(-1)^{\sigma(x_i)} x_1 \tensor \cdots \tensor \overline{\Delta}(x_i)\tensor \cdots \tensor x_q\]
    where the $x_i$ are elements of $(s^{-1}\overline{C})_{n_i}$ and $\sigma(x_i)$ is the Koszul sign $\deg x_1 + \cdots + \deg x_{i-1}$.
\end{defn}
\begin{prop}\label{prop:cobar_bicomplex}
The totalization of $\Cobar^\bullet_\bullet(C)$ given by 
\[\tot_k(\Cobar^\bullet_\bullet) = \bigoplus_{p-q = k} \Cobar^q_p\]
with differential $\partial_{\tot} = d_H+(-1)^qd_V$ on $\Cobar^q_p$, is isomorphic as a dg algebra to $\Cobar(C_\bullet)$.
\begin{proof}
    By the preceeding discussion, we have an equality of groups $\tot_k(\Cobar^\bullet_\bullet)$ and $\Cobar_k(C)$. Now by Proposition \ref{prop:maps_extend}, it suffices to check that $\partial_{\tot}$ extends the map $s^{-1}\overline{C} \to s^{-1}\overline{C} \oplus s^{-1}\overline{C}^{\tensor 2}$ given in the definition of the cobar complex. Indeed, we have that on
    \[s^{-1}\overline{C} = \bigoplus_{n_1 =p\geq 0} (s^{-1}\overline{C})_{n_1} =\bigoplus_{p\geq 0} \Cobar^1_p,\]
    the differential is $\partial_{\tot} = d_H - d_V$ which is exactly the cobar differential.
\end{proof}
\end{prop}

What is miraculous is that this purely algebraic construction encodes a good amount of topological information. This is the content of Adams' theorem:
\begin{thm}[\cite{AdamsCobar}]
    Let $(X,x)$ be a simply connected based topological space. Let $\Omega_xX$ be the space of loops in $X$ based at $x$. Then there is a natural isomorphism
    \[H_\bullet(\Cobar(C_\bullet(X)))\cong H_\bullet(\Omega_xX).\]
\end{thm}
Thus we can regard the cobar construction as the algebraic analogue of the based loop space functor $\Omega_x(-)$. As the prefix \emph{co-} suggests, the cobar construction is adjoint to a functor 
\[\Barr: \catname{dgAlg} \to \catname{dgConilCoalg}\]
called the \emph{bar construction}, which is (roughly) the algebraic version of the delooping of a space. See Chapter 2 of \cite{Loday2012} for precise statements and a comparison of the two constructions.
\begin{rem}
The name \emph{bar} construction comes from typesetting. Pre-\LaTeX, a tensor $x_1 \tensor \cdots \tensor x_n$ would be written $x_1 | \cdots | x_n$. Adams attributes the name \emph{cobar} to H. Cartan (see footnote 3 of \cite{AdamsCobar}).
\end{rem}

\subsection{Two-sided version} 
The cobar construction of a dg coalgebra is a special case of a more general \emph{two-sided} cobar construction. An analogous two-sided bar construction is well documented (for example \cite{May1972}, \cite{Zhang2019TwoSidedBar}, \cite{May2025}). Here we give an exposition of the dual construction, which seems to not appear in the literature.
\begin{defn}
Let $C = \overline{C}\oplus R$ be a coaugmented $R$-coalgebra and let $N,M$ be left and right comodules over $C$, respectively. The \emph{two-sided cobar construction on} $(N, C, M)$ is the cosimplicial $R$-module 
\[\Cobar(N,C,M)^q = M\tensor C^{\tensor q} \tensor N\]
with cofaces $d^i: \Cobar^{q-1}\to \Cobar^q$ given by
\[d^i = \begin{cases}
    \Delta^R \tensor \id^{\tensor q} & i =0  \\
    \id^{\tensor i } \tensor \Delta \tensor \id^{q-i} & i \in \{1, ..., q-1\} \\
    \id^{\tensor q}\tensor \Delta^L & i =q
\end{cases}\]
and codegeneracies $s^i: \Cobar^{q+1}\to \Cobar^q$ given by 
\[s^i = \id^{\tensor i} \tensor \epsilon \tensor \id^{\tensor q-i-1}\]
\end{defn}
If $C,N,M$ have additional dg structure (i.e. $C$ is a dg coalgebra and $N, M$ are dg comodules), then we also get an internal differential on $\Cobar(N,C,M)$ in addition to the differential coming from the coalgebra and comodule maps. Taking these new differentials into account, we obtain the following notion, generalizing the previous discussion of the cobar bicomplex:
\begin{defn}
    Let $(C, \partial_C)$, $(N, \partial_N)$, and $(M, \partial_M)$ be as above. The \emph{two-sided cobar bicomplex} has underlying $R$-module 
    \[\Cobar(N,C,M)^{\bullet}_\bullet = \bigoplus_{q\geq 0} M\tensor C^{\tensor q} \tensor N\]
    and bigrading 
    \[\Cobar(N,C,M)^q_p = \bigoplus_{a + b_1 + \cdots + b_q + c = p} M_a \tensor C_{b_1} \tensor \cdots \tensor C_{b_q}\tensor N_c\]
    with vertical differential given by the alternating sums 
    \[d^V: \Cobar(N,C,M)^{q-1}_\bullet\to \Cobar(N,C,M)^{q}_\bullet,\quad d_V = \sum_{i=0}^q d^i \]
    and horizontal differential coming from the comodule and coalgebra differentials:
    \[d_H: \Cobar(N,C,M)^q_p\to \Cobar(N,C,M)^q_{p-1}\]
    \begin{equation}\label{eqn: twoside_cobar_differential}
    \begin{split}
    d_H(m\tensor c_1 \tensor \cdots \tensor c_q \tensor n) &= \hspace{2pt} \partial_M(m) \tensor c_1 \tensor \cdots \tensor c_q \tensor n \\
    &+\sum_{i=1}^q (-1)^{|m| + \sigma(c_i)} m \tensor c_1 \tensor \cdots \tensor \partial_C(c_i) \tensor \cdots \tensor c_q \tensor n \\
    &+ (-1)^{|m|+ \sigma(c_q)} m \tensor c_1 \tensor \cdots \tensor c_q \tensor \partial_N(n)
    \end{split}
    \end{equation}
    where $\sigma(x_k) = |x_1| + \cdots |x_{k-1}|$ is the Koszul sign. Then totalizing with total degree $p-q$ and differential $d_H + (-1)^qd^V$ on $\Cobar(N,C,M)^q_\bullet$ yields a dg algebra, the \emph{two-sided cobar construction} of $(N,C,M)$.
\end{defn} 

This two-sided construction reduces to the one in the previous subsection as follows. Let $C$ be a dg coalgebra with coaugmentation $\eta: R\to C$. Set $N=M=R$, regarded as a dg $R$-module with $R$ concentrated in degree zero. Define the left action $\Delta^L: N\to C\tensor N$ via 
\[\begin{tikzcd}[cramped]
    % https://q.uiver.app/#q=WzAsMyxbMCwwLCJSIl0sWzAsMSwiQyJdLFsxLDEsIkNcXHRlbnNvciBSIl0sWzAsMSwiXFxldGEiLDJdLFswLDIsIlxcRGVsdGFeTCIsMCx7InN0eWxlIjp7ImJvZHkiOnsibmFtZSI6ImRhc2hlZCJ9fX1dLFsxLDIsIlxcY29uZyIsMl1d
	R & \\
	C & {C\tensor R}
	\arrow["\eta"', from=1-1, to=2-1]
	\arrow["{\Delta^L}", dashed, from=1-1, to=2-2]
	\arrow["\cong"', from=2-1, to=2-2]
\end{tikzcd}\]
and likewise for the right action. This is the correct reduction:
\begin{prop}\label{prop:twoside_reduction}  
$\tot\Cobar(R,C,R)$ as above is quasi-isomorphic to the single cobar construction from Definition 3.1.
\end{prop}
For the proof we need a fact from homological algebra.
\begin{lem}\label{lem:quasiiso_bicomplex}
    Let $f: D_{\bullet, \bullet} \to E_{\bullet, \bullet}$ be a map of first quadrant bicomplexes of abelian groups. If $f$ is a row-wise or column-wise quasi-isomorphism, then the induced map on totalizations $f_*: \tot D\to \tot E$ is a quasi-isomorphism.
    \begin{proof}
        We do the proof where $f$ is a row-wise quasi-isomorphism, i.e. for fixed $q$ we have a quasi-isomorphism 
        \[f: D_{\bullet, q}\to E_{\bullet, q}.\] 
        Let us first consider the truncated bicomplexes $D^{\leq n}_{\bullet,\bullet}, E^{\leq n}_{\bullet,\bullet}$ where the first $n$ rows are nonzero. Then $f_*$ restricts to a map $D^{\leq n}_{\bullet,\bullet}\to E^{\leq n}_{\bullet,\bullet}$. We will induct on $n$. If $n=0$, there is nothing to prove. Suppose now that $n>1$ and $f: D^{\leq n-1}_{\bullet,\bullet}\to E^{\leq n-1}_{\bullet,\bullet}$ induces a quasi-isomorphism of totalizations. 

        We totalize by taking $\tot(D)_k = \bigoplus_{p+q=k} D_{p,q}$. So we have an exact sequence of chain complexes
        \[\tot(D^{\leq n-1}) \hookrightarrow  \tot(D^{\leq n}) \twoheadrightarrow D_{\bullet, n}[-n]\] 
        where $D_{\bullet, n}[-n]$ is the chain complex beginning below in degree 0,
        \[D_{\bullet, n}[-n]: 0\leftarrow \cdots \leftarrow {D^{0,n}} \leftarrow D^{1,n}\leftarrow \cdots\]
        having first nonzero term $D^{0,n}$ in degree $n$. Repeating the same construction for $E$, we get a commutative diagram of short exact sequences 
        \[\begin{tikzcd}[cramped]
            % https://q.uiver.app/#q=WzAsMTAsWzEsMCwiXFx0b3QoRF57XFxsZXEgbi0xfSkiXSxbMSwxLCJcXHRvdChFXntcXGxlcSBuLTF9KSJdLFsyLDAsIlxcdG90KERee1xcbGVxIG59KSJdLFsyLDEsIlxcdG90KEVee1xcbGVxIG59KSJdLFszLDAsIkRfe1xcYnVsbGV0LCBufVstbl0iXSxbMywxLCJFX3tcXGJ1bGxldCwgbn1bLW5dIl0sWzAsMCwiMCJdLFswLDEsIjAiXSxbNCwwLCIwIl0sWzQsMSwiMCJdLFswLDIsIiIsMCx7InN0eWxlIjp7InRhaWwiOnsibmFtZSI6Imhvb2siLCJzaWRlIjoidG9wIn19fV0sWzEsMywiIiwwLHsic3R5bGUiOnsidGFpbCI6eyJuYW1lIjoiaG9vayIsInNpZGUiOiJ0b3AifX19XSxbMiw0LCIiLDAseyJzdHlsZSI6eyJoZWFkIjp7Im5hbWUiOiJlcGkifX19XSxbMyw1LCIiLDAseyJzdHlsZSI6eyJoZWFkIjp7Im5hbWUiOiJlcGkifX19XSxbMCwxLCJmXyoiLDJdLFs0LDUsImZfKiJdLFsyLDMsIiIsMSx7InN0eWxlIjp7ImJvZHkiOnsibmFtZSI6ImRhc2hlZCJ9fX1dLFs2LDBdLFs3LDFdLFs0LDhdLFs1LDldXQ==
            0 & {\tot(D^{\leq n-1})} & {\tot(D^{\leq n})} & {D_{\bullet, n}[-n]} & 0 \\
            0 & {\tot(E^{\leq n-1})} & {\tot(E^{\leq n})} & {E_{\bullet, n}[-n]} & 0
            \arrow[from=1-1, to=1-2]
            \arrow[hook, from=1-2, to=1-3]
            \arrow["{f_*}"', from=1-2, to=2-2]
            \arrow[two heads, from=1-3, to=1-4]
            \arrow[dashed, from=1-3, to=2-3]
            \arrow[from=1-4, to=1-5]
            \arrow["{f_*}", from=1-4, to=2-4]
            \arrow[from=2-1, to=2-2]
            \arrow[hook, from=2-2, to=2-3]
            \arrow[two heads, from=2-3, to=2-4]
            \arrow[from=2-4, to=2-5]
        \end{tikzcd}\]
        in which the two outer arrows are quasi-isomorphisms. The five lemma concludes the induction. Hence $f_*: \tot(D^{\leq n})\to \tot(E^{\leq n})$ is a quasi-isomorphism for all $n$. 
        
        We can now write $D$ and $E$ as the direct limits (which are colimits) of their truncated subcomplexes
        \[D = \varinjlim_{n} D^{\leq n}, \quad E = \varinjlim_{n} E^{\leq n}.\]
        Now totalization is a colimit, and colimits commute with colimits, so $f_*$ assembles into a map of direct limits 
        \[f_*: \varinjlim_n \tot(D^{\leq n}) \to \varinjlim_n \tot(E^{\leq n}).\]
        Since homology commutes with filtered colimits, we a map of colimits
        \[\varinjlim_n H_*\tot(D^{\leq n})\to \varinjlim_n H_*\tot(E^{\leq n})\]
        where each map $H_*\tot(D^{\leq n}) \to H_*\tot(E^{\leq n})$ is an isomorphism, implying that 
        \[f_*: H_*\tot(D)\to H_*\tot(E)\]  
        is too. The proof supposing $f$ is a column-wise quasi-isomorphism proceeds identically, by truncating column-wise.
    \end{proof}
\end{lem}
Because we will use this notion many more times times in this thesis, we make the following definition: 
\begin{defn}
    A map $f: D \to E$ of first quadrant spectral sequences is a \emph{quasi-isomorphism} if it is a row- or column-wise quasi-isomorphism. 
\end{defn}
Then Lemma \ref{lem:quasiiso_bicomplex} says that a quasi-isomorphism of bicomplexes induces a quasi-isomorphism of totalizations. 
\begin{proof}[Proof of Proposition \ref{prop:twoside_reduction}]
    We take the normalized cochain complex on each column of $\Cobar_p^\bullet$ to obtain:
    \begin{equation}
	\begin{split}
		N\Cobar_p^q &= \bigcap_{i=0}^{q-1}\ker(s^i: \Cobar_p^q\to \Cobar_p^{q-1}) \\
		&= \bigcap_{i=0}^{q-1}\ker(\id^i\tensor \epsilon \tensor \id^{\tensor q-i-1})_p \\
		&=  (\ker(\epsilon)^{\tensor q})_p \\
		&= \bigoplus_{m_1 + \cdots + m_q = p} \overline{C_{n_1}} \tensor \cdots \tensor \overline{C_{n_q}} \\
        &= old cobar.
	\end{split}
\end{equation}
Upon checking differentials, we find that the normalized bicomplex $N\Cobar$ is equal to the reduced cobar construction on $C$. Since $N\Cobar \hookrightarrow \Cobar$ is a quasi-isomorphism, we conclude by Proposition \ref{prop:twoside_reduction}.
\end{proof}
% Let us now return to topological problem where we want to have two coaugmentations of $C_\bullet(X)$ corresponding to the inclusion of distinct basepoints $\iota_a: \{a\} \hookrightarrow X$ and $\iota_b: \{b\} \hookrightarrow X$. We regard $C_\bullet(\{a\}) \cong \Z$ as a right $C_\bullet(X)$-comodule via the following composite
% % https://q.uiver.app/#q=WzAsMyxbMCwwLCJDX1xcYnVsbGV0KFxce2FcXH0pIl0sWzEsMSwiQ19cXGJ1bGxldChcXHthXFx9KVxcdGVuc29yIENfXFxidWxsZXQoWCkiXSxbMSwwLCJDX1xcYnVsbGV0KFxce2FcXH0pXFx0ZW5zb3IgQ19cXGJ1bGxldChcXHthXFx9KSJdLFswLDIsIlxcQVdfe1xce2FcXH19Il0sWzIsMSwiXFxpZFxcdGVuc29yIFxcaW90YV9hIl0sWzAsMSwiXFxEZWx0YV5SIiwyLHsic3R5bGUiOnsiYm9keSI6eyJuYW1lIjoiZGFzaGVkIn19fV1d
% \[\begin{tikzcd}
% 	{C_\bullet(\{a\})} & {C_\bullet(\{a\})\tensor C_\bullet(\{a\})} \\
% 	& {C_\bullet(\{a\})\tensor C_\bullet(X)}
% 	\arrow["{\AW_{\{a\}}}", from=1-1, to=1-2]
% 	\arrow["{\Delta^R}"', dashed, from=1-1, to=2-2]
% 	\arrow["{\id\tensor \iota_a}", from=1-2, to=2-2]
% \end{tikzcd}\]
% and likewise for $C_\bullet(\{b\})$ as a left $C_\bullet(X)$-comodule. 
Finally, we should check that this generalization of the cobar construction extends Adams' theorem.
\begin{thm}\label{thm:adams_path}
    Let $X$ be a space with basepoints $a,b$. Regard $C_\bullet(\{a\})$ and $C_\bullet(\{b\})$ as ... 
    
    There is a natural quasi-isomorphism from the totalization of the two-sided cobar construction 
    \[\tot\Cobar(C_\bullet(\{b\}), C_\bullet(X), C_\bullet(\{a\}))\]
    to the singular chains on the path space $C_\bullet(\Omega_{a,b}X)$.
    \begin{proof}
    
    \end{proof}
\end{thm}

\subsection{Homology and the fundamental group}
Adams' theorem allows one to use homology and the cobar construction to approximate the fundamental group:
\begin{thm}[\cite{Stallings1975}, Theorem 5.4]\label{thm:stallings}
    Let $R$ be a ring and $(X,x)$ a path-connected space. Let $\mathcal{I}$ be the augmentation ideal of the group ring $R[\pi_1(X,x)]$. Let $E$ be the spectral sequence associated to the cobar construction on $C_\bullet(X)$ with basepoint $x$. Then there is an isomorphism
    \[\mathcal{I}^p/\mathcal{I}^{p+1} \cong E^\infty_{-p,p}.\]
\end{thm}
A dual version of this theorem with cohomology and the bar construction can be found in \cite{gadish_letterbraiding}, Theorem 1.2(3). The goal of this subsection will be to prove an integral version of this theorem for the two-sided cobar construction.
\begin{thm}\label{thm:stallings_generalization}
    Let $X$ be a path-connected space with basepoints $a,b$. Let $\mathcal{I}$ be the augmentation ideal of the group ring $\Z\pi_1(X,a)$. Let $E$ be the spectral sequence associated to the two-sided cobar construction 
    \[\Cobar(NC_\bullet(\{b\}), NC_\bullet(X), NC_\bullet(\{a\}))\]
    Then there is an isomorphism
    \[\mathcal{I}^p/\mathcal{I}^{p+1} \to E^\infty_{-p,p}.\]
    \begin{proof}
    It will again to turn out to be easier to work with the normalized cobar, a la the proof of Proposition \ref{prop:twoside_reduction}. So abbreviate $N\Cobar(\cdots)$ by $\Cobar$. By Theorem \ref{thm:adams_path}, we have an isomorphism 
    \[H_0(\tot(\Cobar)) \to H_0(\Omega_{a,b}X) \cong \Z\pi_0(\Omega_{a,b}X).\]
    Recall that the bigrading on the two-sided cobar construction is given by 
    \[\Cobar(N,C,M)^q_p = \bigoplus_{a + b_1 + \cdots + b_q + c = p} M_a \tensor C_{b_1} \tensor \cdots \tensor C_{b_q}\tensor N_c\] 
    so that the degree zero totalization is given exactly by terms with $p=-q$. Since $NC_\bullet(\{a\})$ and $NC_\bullet(\{b\})$ have only nonzero terms concentrated in degree zero, while in the reduced complex we have $\overline{NC_0(X)} = 0$, we are forced to have 
    \[\Cobar^p_{-p} = NC_\bullet(\{a\})\tensor \overline{NC_1(X)}^{\tensor p}\tensor NC_\bullet(\{b\}).\]
    We can then define a map from $\tot_0(\Cobar)$ to a rank one free module over $\Z\pi_1(X,a)$ as follows. Fix a spanning tree of $X$ with basepoint at a, i.e. a collection of paths for each $x$ in $a$
    \[\{s: [0,1]\to X: s_x(0) = a, s_x(1) = x\}.\]
    with the requirement that $s_a$ is the constant path at $a$. Then define $\tilde{\phi}: \tot_0(\Cobar)\to \Z\pi_1(X,a)[s_b]$ via 
    \[\tilde{\phi}(1_a \tensor \sigma_1 \tensor \cdots \sigma_p \tensor 1_b) = (1_a-\tilde{\sigma}_1)\cdots (1_a-\tilde{\sigma}_p)s_b\]
    where 
    \[\tilde{\sigma}_i = s_{\sigma_{i}(0)} * \sigma_i * s_{\sigma_i(1)}.\]
    Clearly the image of $\tilde{\phi}$ is all of $\mathscr{I}^p$. We claim that $\tilde{\phi}$ descends to $H_0\tot(\Cobar)$.
    \end{proof}
\end{thm}
% \begin{lem}[\cite{Stallings1975}, Corollaries 5.2, 5.3]
%     Let $X$ be a path-connected space. Something about two-sided cobar giving same thing for $X$ and $K(\pi_1(X), 1)$.
% \end{lem}

% \begin{thm}[\cite{Rivera2022}, Theorem 1]
%     Let $(X,x)$ be a path connected based topological space. Let $\Omega_xX$ be the space of loops in $X$ based at $x$. Then $\Cobar(C_\bullet(X))$ is quasi-isomorphic to $C_\bullet(\Omega_xX)$ as dg algebras. 
% \end{thm}
% \begin{rem}
%     We have stated the theorem for singular chains $C_\bullet(X)$ rather than some modified chains $C_\bullet(X,b)$ (not the relative chains!), but these are quasi-isomorphic? 
% \end{rem}
% \begin{rem}
% Cobar not preserving quasi isomorphisms
% \end{rem}
% Moreover, there is a canonical way in which the cobar construction is related to the augmentation ideal-adic truncations of the fundamental group ring:
% \begin{thm}[\cite{gadish_letterbraiding}, Theorem 1.2(3)]
%     Let $A$ be a PID, and $\Gamma$ a group. Let $\mathcal{I}$ be the augemntation ideal of the group ring $A[\Gamma]$. Then we have the following isomorphism 
%     \[(\mathcal{I}^n)^{\perp} \cong H^0(\Barr_{<n}(C^\bullet(B\Gamma)); A)\]
%     where the left hand side is the orthogonal complement of $\mathcal{I}^n < A[\Gamma]$, and the left hand side is the zeroth cohomology of the subcoalgebra of the bar construction on the cochains of the classifying space $B\Gamma$, consisting of weight $<n$ tensors. \hfill \qed
% \end{thm}
% Taking $A=\Z$, $\Gamma = \pi_1(X,x)$, and dualizing, we obtain the following:
% \begin{cor}\label{cor:gadish}
%     There is an isomorphism of groups
%     \[\Z\pi_1(X,x)/ \mathcal{I}^{n+1} \cong H_0(\Cobar_{\leq n}(C_\bullet(B\pi_1(X,x))))\]
%     and consequently 
%     \[\Z\pi_1(X,x)/ \mathcal{I}^{n+1} \cong H_0(\Cobar_{\leq n}(C_\bullet(X)))\]
%     where $\Cobar_{\leq n}$ denotes the cobar complex with length $>n$ tensors truncated.
%     \begin{proof}
%         The first statement follows immediately from Gadish's theorem. The second statement follows since the zeroth homology of the cobar construction depends only on the fundamental group of a space. To see this,
%     \end{proof}
% \end{cor}