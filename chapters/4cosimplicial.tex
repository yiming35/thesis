\section{(Co)simplicial objects}
\epigraph{\textit{A cosimplicial object of a category $\mathcal{C}$ could be defined simply as a simplicial object of the opposite category $\mathcal{C}^\mathrm{op}$. This is not really how the human brain works...}}{---Stacks Project, \cite[\href{https://stacks.math.columbia.edu/tag/016I}{Tag 016I}]{stacks-project}}

In this chapter we define cosimplicial objects, the totalization of a cosimplicial space, and provide some examples. We will state the Dold--Kan correspondence and use it to prove a useful statement for later on. Finally we conclude with an exposition of the Alexander--Whitney map, which couldn't find a home in any other chapter. We assume some familiarity with simplicial objects.

\subsection{Definitions} The \emph{simplex category}, denoted $\mathbf{\Delta}$, is the category with
\begin{enumerate}
    \item objects: finite nonempty totally ordered sets. We write $[n]$ for the set $\{0<1<\cdots < n\}$,
    \item morphisms: order-preserving maps, i.e. if $i\leq j$ then $f(i)\leq f(j)$.
\end{enumerate}  
Then a \emph{cosimplicial object} in a category $\mathcal{C}$ is a functor $X^\bullet: \mathbf{\Delta}\to \mathcal{C}$. We denote the image of $[n]$ under $X^\bullet$ by $X^n$. A morphism of two simplicial objects $X^\bullet, Y^\bullet: \mathbf{\Delta}\to \mathcal{C}$ is a natural transformation of functors, i.e. morphisms $X^n\to Y^n$ for all $n$ that commute with morphisms in $\mathbf{\Delta}$. We will denote the category of of cosimplicial objects in $\mathcal{C}$ by $\catname{cs}\mathcal{C}$. In this thesis we will concern ourselves with the cases where $\mathcal{C} = \catname{Top}$ or $\mathcal{C} = \catname{Set}$.
\begin{rem}
Even though simplicial objects are \textit{contravariant} functors, the convention is to denote the simplicies using subscripts, e.g. $X_n$. Conversely, even though simplicial spaces are \textit{covariant} functors, the convention is to denote the cosimplicies with superscripts. Maybe the presence of the suffix co- explains this.
\end{rem}
Recall that the morphisms in the simplex category are generated by two distinguished classes of maps. For $n\geq 1$ and $j\in [n]$, we have injections $\delta_j: [n-1] \to [n]$ where $\delta_j$ skips $j \in [n]$. For $n\geq 0$ and $j\in [n]$, we have $n+1$ surjections $\sigma_j: [n+1]\to [n]$ where $\sigma_j$ sends both $j, j+1$ to $j \in [n]$. For a cosimplicial object $X^\bullet: \mathbf{\Delta} \to\mathcal{C}$, we call the images of the $\delta_j$ \textit{coface maps}, usually denoted $d^j$, and the images of the $\sigma_j$ \textit{codegeneracy maps}, usually denoted $s^j$. So to specify a cosimplicial object in $\mathcal{C}$ it also suffices to list a sequence of objects $X^n \in \mathcal{C}$ for $n\geq 0$, as well as coface and codegeneracy maps satisfying the following \emph{cosimplicial identities}:
\begin{enumerate}
      \item If $i<j$, then $d^j \circ d^i = d^i \circ d^{j-1}$.
      \item If $i<j$, then $s^j \circ d^i = d^i \circ s_{j-1}$.
      \item $\id = s^j \circ d^j = s^j \circ d^{j+1}$.
      \item If $i > j+1$, then $s^j \circ d^i = d^{i-1}\circ s^j$.
      \item If $i\leq j$, then $s^j \circ s^i = s^i \circ s^{j+1}$.
\end{enumerate}
One should think of a cosimplicial object $X^\bullet: \mathbf{\Delta}\to \mathcal{C}$ as a diagram 
\[\begin{tikzcd}[ampersand replacement=\&]
X^0\arrow[r, COSIMPLICIALaltstackar=3] \& 
X^1\arrow[r, COSIMPLICIALaltstackar=5] \&
X^2\arrow[r, COSIMPLICIALaltstackar=7] \&\cdots
\end{tikzcd}\]
where the rightward pointing arrows are the coface maps and the leftward pointing arrows are the codegeneracy maps. In general it helps to think of the coface maps as ``duplicating a coordinate'' and the codegeneracy maps as ``forgetting a coordinate.'' We will see this in the following examples. 
\begin{exmp}[The topological simplicies]
Define the functor $\Delta^\bullet: \mathbf{\Delta} \to \catname{Top}$ which sends $[n]$ to the topological $n$-simplex:
\[\Delta^n = \{(x_1, ..., x_n) \in \R^{n}:  0\leq x_1 \leq \cdots \leq x_n \leq 1\}.\]
Then the coface maps $\Delta^{n-1}\to \Delta^n$ are the inclusions of faces, where $d^j$ is the inclusion of the face opposite the $j$th vertex. The codegeneracy map $s^j$ collapses the line joining the $j$th and $j$th vertex. In coordinates, we have
\[d^j(x_1, ..., x_n) = (x_1, ..., x_j, x_j, ..., x_{n-1}),\]
\[s^j(x_1, ..., x_{n}) = (x_1, ..., \widehat{x_j}, ..., x_n)\]
where $\widehat{\phantom{-}}$ indicates omission. 
\end{exmp}

\begin{exmp}[Path spaces]\label{example:cosimplicial_path_space}
Let $X$ be a topological space with $a,b \in X$. Define the cosimplicial space $P_{a,b}^\bullet X$ whose cosimplicies are 
\[P_{a,b}^0 X = \{*\}, \quad P_{a,b}^nX = X^n \text{ for $n\geq 1$.}\]
The coface maps $d^j: P_{a,b}^{n-1}X\to P_{a,b}^{n}$ are given by
\[d^j(x_1, ..., x_{n-1}) = \begin{cases}
  (a, x_1, ..., x_{n-1}) & j=0\\
  (x_1, ..., x_j, x_j, ..., x_{n-1}) & j \in \{1,..., n-1\}\\
  (x_1, ..., x_{n-1}, b) & j=n
\end{cases}\]
The codegeneracy maps $s^j: P_{a,b}^{n+1}X \to P_{a,b}^{n}X$ are given by
\[s^j(x_1, ..., x_{n+1}) =(x_1, ...,\widehat{x_{j+1}}, ... x_{n+1}), \quad j \in \{0, ..., n\}\]
where the $\widehat{\phantom{-}}$ denotes omission.
\end{exmp}
We leave it to the reader to verify that the maps in both examples satisfy the cosimplicial identities. Why we refer to Example \ref{example:cosimplicial_path_space} by \textit{path spaces} will become evident in the next section. 

\subsection{Totalization} 
Cosimplicial spaces provide a useful model for many types of topological spaces, including the based path and loop spaces. This is done via \emph{totalization}, which is dual to the notion of geometric realization of a simplicial set. 

Given a cosimplicial space $X^\bullet: \mathbf{\Delta}\to \catname{Top}$, define the \emph{totalization} of $X^\bullet$ to be the space of maps from the cosimplicial simplices to $X^\bullet$:
\[\tot(X^\bullet) := \Hom_{\catname{csTop}}(\Delta^\bullet, X^\bullet),\]
i.e. maps $f^n: \Delta^n \to X^n$ for all $n\geq 0$ that commute with the coface and codegeneracy maps. We topologize it as a subspace of $\prod_{n\geq 0} \Hom(\Delta^n, X^n)$ with the compact-open topology. Thus totalization gives us a functor from $\catname{csTop}$ to $\catname{Top}$. 

What seems to be happening here is that by iterating the face maps, we are creating finer and finer piecewise subdivisons of paths whose endpoints are at $a$ and $b$. Indeed, 
\begin{prop}
  $\tot(P_{a,b}^\bullet X)$ is homeomorphic to the path space $\Omega_{a,b}X$ of paths in $X$ beginning at $a$ and ending at $b$.
  \begin{proof}
    A point of $\tot(P_{a,b}^\bullet X) = \hom_{\catname{csTop}}(\Delta^\bullet, P_{a,b}^\bullet X)$ us a sequence of continuous maps 
    \[f= \{f_i: \Delta^i \to X^i\}_{i \geq 0}\]
    commuting with the coface and codegeneracy maps. Fix $n\geq 2$ and $k \in \{1, ..., n\}$. Consider the following composition of codegeneracy maps
    \[\alpha_{n,k} :=  \underbrace{s^{n-1} \circ s^{n-2} \circ \cdots \circ s^{k-2} \circ s^k\circ \cdots \circ s^0}_{\text{$n-1$ maps}}\]
    where we compose all the degeneracies except for $s^{k-1}$. This gives us a map $\Delta^n\to \Delta^1$ and likewise for $X^n\to X^1$. Then for $f  = \{f_0, f_1, ...\} \in \tot(P_{a,b}^\bullet X)$, we have by commutativity that 
    \[f_1 \circ \alpha_{n,k}  = \alpha_{n+1,k}\circ f_n.\]
    But now the right hand side is just picking out the $k$th coordinate of $f_n$. Hence for $n \geq 2$, $f_n$ is completely determined by $f_1$, so that the projection $\Phi: \tot(P_{a,b}^\bullet X)\to \Map(\Delta^1, X)$ given by $\{f_i\}_{i\geq 0} \mapsto f_1$ is injective.

    We claim next that $\Phi$ is actually a map into $\Omega_{a,b}X\subset \Map(\Delta^1, X)$. The cosimplicial relations imply
    \[f_1\circ d^0 = d^0 \circ f_0 = \const_a, \quad f_1 \circ d^1 = d^1\circ f_0 = \const_b\]
    so $f_1(0) = a$ and $f_1(1) = b$ as desired. 
    
    Lastly we define an inverse to $\Phi: \tot(P^\bullet_{a,b}X) \to \Omega_{a,b}X$. For a given path $\gamma \in \Omega_{a,b}X$ consider the family of maps $\{f_i: \Delta^i \to X^i\}_{i\geq 0}$ given as follows. We let $f^0$ be the constant map, $f^1 = \gamma$, and for $n \geq 2$ define
    \[f_n(x_1, ..., x_n) = (\gamma(x_1), \gamma(x_1+x_2), ..., \gamma(x_1 + \cdots + x_n)).\]
    Clearly this is an inverse to $\Phi$. We leave it to the reader to check that the family of maps $\{f_i\}$ commutes with the coface and codegeneracies, and that $\Phi$ and its inverse are continuous maps. 
  \end{proof}
\end{prop}
We will call $P_{a,b}^\bullet X$ a cosimplicial model for the path space of $X$. When $a=b$ we get a cosimplicial model for the based loop space of $X$. It turns out that the cosimplicial space $P_{x,x}^\bullet X$ is in some sense the underlying cosimplicial set of the cobar construction of $C_\bullet(X)$, as we will see in the course of this thesis.

\subsection{Dold--Kan} Recall that the standard Dold--Kan correspondence gives an equivalence of categories between simplicial abelian groups and (non-negative) chain complexes of abelian groups. We'll restate some important results and their duals. Although these are duals, there is once again some asymmetry, as suggested by the quote.
\begin{defn}
  Let $A$ be a simplicial abelian group. Its \emph{normalized chain complex} $(NA_\bullet, \partial)$ has
  \[NA_n = \bigcap_{i=1}^{n} \ker(d^i: A_n \to A_{n-1}),\quad \partial = d_0: NA_n \to NA_{n-1}\]
  and its \emph{Moore complex} $(MA_\bullet, \partial')$ has 
  \[MA_n = A_n, \quad \partial' = \sum_{i=0}^n (-1)^i d_i: A_n\to A_{n-1}.\]
\end{defn}

The following results and their proofs can be found in \cite{Goerss2009}, Chapter III.2:
\begin{thm}[Dold--Kan correspondence] Let $A$ be a simplicial abelian group. Then:
  \begin{enumerate}
  \item The functor $N: \catname{sAb} \to \catname{Ch_+(Ab)}$ is an equivalence of categories between simplicial abelian groups and non-negative chain complexes of abelian groups.
  \item The inclusion of chain complexes $NA\hookrightarrow MA$ is a chain homotopy equivalence, natural in $A$.
  \item There is a functorial direct sum decomposition $MA = NA \oplus DA$, where $DA$ is the subcomplex generated by the images of all the degeneracy maps. Consequently, $DA$ is acyclic.
  \end{enumerate}
\end{thm}


Now we dualize. Recall that for simplicial objects, the face maps $d_i$ lower degree. For cosimplicial objects, the face maps $d^i$ raise degree, so we should get an equivalence between cosimplicial abelian groups and cochain complexes of abelian groups. This is indeed what happens.
\begin{defn}
  Let $C^\bullet$ be a cosimplicial abelian group. Its normalized cochain complex $(NC^\bullet, \partial)$ has 
  \[NC^n = \coker\bigoplus_{i=1}^{n} (d^i: C^{n-1}\to C^n), \quad \partial = d^0: NC^n \to NC^{n+1}\]
  and its Moore complex has 
  \[MC^n = C^n, \quad \partial' = \sum_{i=0}^{n+1} (-1)^i d^i: C^n\to C^{n+1}.\]
\end{defn}
Analogously we obtain the following statements:
\begin{thm}[Dold--Kan correspondence, dual] Let $C$ be a cosimplicial abelian group.
  \begin{enumerate}
    \item The functor $N: \catname{csAb} \to \catname{coCh_+(Ab)}$ is an equivalence of categories between cosimplicial abelian groups and non-negative cochain complexes of abelian groups.
    \item The quotient of cochain complexes $MC \rightarrow NC$ is a cochain homotopy equivalence, natural in $C$.
    \item There is a functorial direct sum decomposition $MC = NC \oplus DC$, where $DC$ is the subcomplex generated by the images of all the coface maps. Consequently, $DC$ is acyclic.
  \end{enumerate}
\end{thm}
Really we could have stated the previous four theorems for simplicial objects in any abelian category. 
In the following discussion, we replace ``abelian group'' with ``$R$-module'' for some commutative ring $R$. Consider the free $R$-module functor $R[-]:\catname{Set} \to \catname{Mod}_R$. Then given any cosimplicial set $X^\bullet: \mathbf{\Delta} \to \catname{Set}$ the composition of functors $R[X^\bullet]$ yields a cosimplicial $R$-module. We first state a useful lemma, which was taken by a comment of Tom Goodwillie on a MathOverflow thread \cite{Goodwillie} related to the forthcoming Proposition \ref{prop:goodwillie_cor}.
\begin{lem}[Goodwillie's lemma]\label{lem:goodwillie}
    Let $n>0$ and choose $x,y \in X^n$. If $x,y$ are not in the image of any coface maps $d^j: X^{n-1}\to X^n$, then $d^ix = d^jy$ implies $i=j$ and $x=y$.
    \begin{proof}
      First suppose $i<j$. Then $d^ix = d^jy$ implies 
      \[d^i s^{j-1}x = s^jd^ix = s^jd^jy = y\]
      so that $y$ is in the image of a coface map, a contradiction. The same argument for $j<i$ leaves only the possibility $i=j$. But $s^id^i =\id$, implying that $d^i$ is injective, so $x=y$.
    \end{proof} 
\end{lem}
\begin{prop}\label{prop:goodwillie_cor}
  Under the Dold--Kan correspondence, the normalized cochain complex of $R[X^\bullet]$ has zero cohomology in positive degree.
  \begin{proof}
  Let $NR[X^\bullet]$ be the normalized cochain complex of $R[X^\bullet]$. In each degree, we have 
  \[NR[X^n] = R[X^n] / D^n\]
  where $D^n$ is the free module generated by $\cup_{i=1}^n \im d^i$. It follows that $NR[X^n]$ is the free module on the set $U^n:=X^n \backslash \cup_{i=1}^n \im d^i$. Let $A_n$ be the subset of $X^n$ consisting of elements not in the image of any coface map, and define 
  \[B^n = \{d^0x: x \in A^{n-1}, d^0x \notin \cup_{i=1}^n d^i\}.\]
  We make the following two claims:
  \begin{enumerate}
    \item For $n\geq 1$, $U^n = A^n \sqcup B^n$ 
    \begin{proof}
      Disjointness and the inclusion $A^n\sqcup B^n \subset U^n$ are clear. Conversely, let $x \in U^n$. If $x \notin A^n$, then $x = d^k(y)$ for some $y \in X^{n-1}$, in fact $k=0$ by definition of $U^n$. It remains to show $y \in A^{n-1}$. If $y \notin A^{n-1}$ then $y = d^l(z)$ for some $z \in A^{n-2}$ (here if $n=1$ this is impossible). So $x = d^0d^l(z)$, and applying a cosimplicial identity we have $x = d^{l+1}d^0(z)$, contradicting the fact that $x \in U^n$. This completes the proof of the claim.
    \end{proof}
    \item For $n\geq 2$, $B^n = \im(d^0: X^{n-1}\to X^n)$.
    \begin{proof}
      It suffices to show that if $x \in \im d^0$, then $x \notin \cup_{i=1}^n \im d^i$. Suppose for contradiction that 
      \[x = d^0a = d^jb\]
      for $a \in A^{n-1}$, $b \in X^{n-1}$, and $j\geq 1$. Applying $s^0$ to both sides we obtain
      \[a = s^0d^0a = s^0d^jb = \begin{cases}
      b & j=1 \\
      d^{j-1}s^0b & j>1
      \end{cases}.\]
      In the first case we have $d^0a = d^1b$, contradicting Goodwillie's lemma since $a=b \in A^{n-1}$. In the second case we see that $a$ is in the image of a coface map $d^{j-1}$, a contradiction. This completes the proof.
    \end{proof}
  \end{enumerate}
  Now we compute the coface map 
  \[d^0: A^n \sqcup B^n \to A^{n+1}\sqcup B^{n+1}.\]
  By the two claims, $d^0$ is an bijection $A^n \to B^{n+1}$. On $B^n$, we have $d^0x = d^0d^0y = d^1d^0y$ for some $y \in X^{n-1}$ by cosimplicial identities, so this is zero in $NC^{n-1}$. It follows that 
  \[H^n(R[X^\bullet]) = \frac{\ker (d^0: NC[X^n] \to NC[X^{n+1}])}{\im(d^0: NC[X^{n-1}]\to NC[X^n])} = \frac{B^n}{B^n} = 0\]
  completing the proof.
  \end{proof}
\end{prop}
\begin{rem}
  Proposition \ref{prop:goodwillie_cor} is completely false for simplicial sets which carry a lot of homological information! For example, gien a simplicial set $K_\bullet$, the chain complex $NR[K_\bullet]$ computes the homology of $K$. If these groups were all zero we would be in trouble.
\end{rem}

%  By ..., we may write the normalized chain groups
%       \[NC^n = \coker \bigoplus_{i=1}^n (d^i: R[X^n]\to R[X^{n}])\]
%       as the direct sum 
%       \[R[\tilde{X}^n] \oplus (\im d^0).\] 
%       The differential $d^0$ is zero on the second summand since $d^0d^0 = d^1d^0$, which gets quotiented out, and the differential is an isomorphism from $R[\tilde{X}^n]$ into the $\im d^0$ summand. Hence for $n>0$,
%       \[H^n(R[X^\bullet]) = \frac{\ker \partial_n}{\im \partial_{n-1}} = \frac{\im d^0}{\im d^0} = 0.\]

% \begin{lem}\label{lem:goodwillie_cor}
%     Any $x \in X^n$ can be written in the following form:
%     \[x = d^{i_k}\circ  \cdots\circ d^{i_1}(x_k)\]
%     for $x_k \in X^{n-k}$ not in the image of any coface map and $i_1 < \cdots < i_k$. Moreover, this form is unique.
%     \begin{proof}

%     Let $\tilde{X}^n\subset X^n$ be the subset of cosimplicies not in the image of any coface map. If $x \in \tilde{X}^n$, we are done. Otherwise, $x = d^{i_1}(x_1)$ for some $x_1 \in X^{n-1}$.
%     If $x_1 \in \tilde{X}^{n-1}$, we are done. Otherwise continue iterating to obtain 
%     \[x = d^{i_k}\circ  \cdots\circ d^{i_1}(x_k).\]
%     for $x_k \in \tilde{X}^{n-k}$.
%     Now applying the cosimplicial identity \cite[\href{https://stacks.math.columbia.edu/tag/016I}{Tag 016I}]{stacks-project}
%     \[a \leq b \implies d^a\circ d^b = d^{b+1}\circ d^a \]
%     starting with $a = i_2, b=i_1$, we can continue flipping the indices to obtain $i_1 < \cdots < i_k$.

%     For uniqueness, suppose
%     \begin{equation}\label{eqn:uniqueness_proof}
%     x= d^{i_k}\circ  \cdots\circ d^{i_1}(x_k) = d^{j_l}\circ  \cdots\circ d^{j_1}(y_l)
%     \end{equation}
%     such that $x_k \in \tilde{X}^{n-k}$, $y_l \in \tilde{X}^{n-l}$, and $i_1 < \cdots i_k$ and $j_1 < \cdots j_l$.
    
%     % We now proceed by induction on $n$. If $n=0$, there is nothing to show. Suppose for now that $i_1=j_1$. Applying $s^{i_1}$ to both sides of \ref{eqn:uniqueness_proof}, we use the cosimplicial identity  
%     % \[a> b+1 \implies s^b\circ d^a = d^{a-1} \circ s^b\]
%     % to conclude
%     % \[s^{i_1}(x) = d^{i_k-1}\circ \cdots \circ s^{i_1}\circ d^{i_1}(x_k) = d^{j_l-1}\circ  \cdots\circ s^{i_1}\circ d^{j_1}(x_l)\]
%     % Since $s^a\circ d^a = \id$ and $i_1 = j_1$, we have
%     % \begin{equation}\label{eqn:uniqueness_proof_2}
%     %   s^{i_1}(x) = d^{i_k-1}\circ \cdots \circ d^{i_2-1}(x_k) = d^{j_l-1}\circ  \cdots \circ d^{j_2-1}(x_l)
%     % \end{equation}
%     % which by induction implies that $k=l$, $i_j = j_l$, as desired. 

%     % It remains to show that $i_1=j_1$. 
%     \end{proof} 
% \end{lem}

\subsection{Alexander--Whitney}
In the proof of Looijenga's theorem there is a step where we roughly need to move from singular chains on a product space to the product of chains. There is a canonical way to do so, called the Alexander--Whitney map. We will present it in generality before specializing to our case.

In some sense, the Alexander--Whitney map is a way of moving between the category of simplicial abelian groups and the category of chain complexes of abelian groups that preserves their respective monoidal structure. Let us recall these structures. The monoidal product on $\catname{sAb}$ is given levelwise: for $A, B\in\catname{sAb}$, we have 
\[(A\tensor B)_n = A_n\tensor B_n\]
and the face and degeneracy maps are inherited naturally. The monoidal product on the category of chain complexes is given as follows: for $X,Y \in \catname{Ch}_+(\catname{Ab})$, define 
\[(X\tensor Y)_n = \bigoplus_{p+q = n} X_p\tensor Y_q.\]
and the new differential $(X\tensor Y)_n\to (X\tensor Y)_{n-1}$ by the Koszul sign rule:
\[\partial_{X\tensor Y}(x\tensor y) = \partial_X x\tensor y + (-1)^{\deg x}x \tensor\partial_Yy.\] 
\begin{defn}
Let $M: \catname{sAb} \to \catname{Ch}_{+}(\catname{Ab})$ be the Moore complex functor. The \emph{Alexander--Whitney} map is a natural transformation of the functors $\catname{sAb}\times \catname{sAb} \to \catname{Ch}_+(\catname{Ab})$
\[\AW: M(-\tensor -) \implies M(-)\tensor M(-)\]
whose components are given as follows. For $A, B \in \catname{sAb}$ and $a \in A_n, b \in B_n$, define the map 
\[\AW_{A,B}: M(A\tensor B) \to M(A)\tensor M(B), \quad a\tensor b\mapsto \bigoplus_{p+q = n} \tilde{d}^p(a)\tensor d^q(b)\]
where $d^p$ is the face map induced by $[p]\to [n], i\mapsto i$, and $d^q$ is the face map induced by $[q]\to [n], i \mapsto i+p$. 
\end{defn}
\begin{rem}
$\AW$ restricts to a natural transformation 
\[N(-\tensor -)\implies N(-) \tensor N(-)\]
where $N$ is the normalized chains functor. (We will denote this restriction also by $\AW$.)
\end{rem}
\begin{prop}\label{prop:AW_quasiiso}
Let $A, B \in \catname{sAb}$. Then 
\[\AW_{A,B}: N(A\tensor B) \to N(A)\tensor N(B)\]
is a natural quasi-isomorphism.
\begin{proof}
Kerodon \cite[\href{https://kerodon.net/tag/00S0}{Tag 00S0}]{kerodon}.
\end{proof}
\end{prop}
This is a significant fact in homological algebra; now we employ only a fraction of its power. Let $X, Y$ be a topological space, and $C_\bullet(-)$ denote the singular chains functor from $\catname{Top}$ to $\catname{sAb}$. Then $MC_\bullet(X)$ is the singular chain complex of $X$ in the usual sense.
\begin{cor}
There is a natural quasi isomorphism   
\[MC_\bullet(X\times Y) \to MC_\bullet(X)\tensor MC_\bullet(Y).\]
\begin{proof}
Recall that $C_\bullet(-)$ is the composition of the following functors: 
\[\catname{Top} \xrightarrow{X\mapsto \Map(\Delta^\bullet, X)} \catname{sSet} \xrightarrow{A_\bullet \mapsto \Z[A_\bullet]} \catname{sAb}\]
and under this composition we get 
\[X\times Y\ \mapsto \Map(\Delta^\bullet, X\times Y) = \Map(\Delta^\bullet, X)\times \Map(\Delta^\bullet, Y)  \mapsto \Z[\Map(\Delta^\bullet, X)]\tensor \Z[\Map(\Delta^\bullet, Y)]\]
because the free abelian group functor sends Cartesian products of sets to tensor products of groups. So the simplicial abelian group $C_\bullet(X\times Y)$ is equal to $C_\bullet(X)\tensor C_\bullet(Y)$. Thus applying Proposition \ref{prop:AW_quasiiso} and the Dold--Kan correspondence we are done.
\end{proof}
\end{cor}
Really what we have defined here is a topological Alexander--Whitney map which is a natural transformation of the functors $\catname{Top} \times \catname{Top}\to \catname{sAb}$:
\[MC_\bullet(-\times -)\to MC_\bullet(-)\tensor MC_\bullet(-)\]
For topological spaces $X,Y$, denote the constituent map by $\AW_{X,Y}$.
From now we will use $C_\bullet(X)$ to mean both the singular chain \textit{complex} and the simplicial abelian group, depending on context. 
\begin{cor}\label{cor:topological_aw}
For $q\geq 0$ and a topological space $X$, there is a natural quasi-isomorphism of chain complexes 
\[C_\bullet(X^q) \to C_\bullet(X)^{\tensor q}.\]
\begin{proof}
Iterate the topological Alexander--Whitney map 
\[C_\bullet(X\times X^{q-1}) \xrightarrow{\AW_{X, X^{q-1}}} \cdots \xrightarrow{\id^{\tensor q-2}\tensor \AW_{X, X}} C_\bullet(X)^{\tensor q}.\]
Since each component is a natural quasi-isomorphism, so is the composition.
\end{proof}
\end{cor}
