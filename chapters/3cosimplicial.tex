\section{Cosimplicial spaces}
\epigraph{\textit{A cosimplicial object of a category $\mathcal{C}$ could be defined simply as a simplicial object of the opposite category $\mathcal{C}^\mathrm{op}$. This is not really how the human brain works...}}{---Stacks Project, 14.5 \cite{stacks-project}}

\begin{rem}
Even though simplicial sets are \textit{contravariant} functors, the convention is to denote the simplicies using subscripts, e.g. $X_n$. Conversely, even though simplicial spaces are \textit{covariant} functors, the convention is to denote the cosimplicies with superscripts. Perhaps the presence of the suffix co- explains this.
\end{rem}

\subsection{Path and loop spaces} 

\subsection{Homotopy} ``... there is no interesting homotopy theory of cosimplicial sets!'' concludes a  MathOverflow answer \cite{Goodwillie} of Tom Goodwillie. This section will be devoted to understanding this comment.