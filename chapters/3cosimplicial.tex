\section{Cosimplicial objects}
\epigraph{\textit{A cosimplicial object of a category $\mathcal{C}$ could be defined simply as a simplicial object of the opposite category $\mathcal{C}^\mathrm{op}$. This is not really how the human brain works...}}{---Stacks Project, 14.5 \cite{stacks-project}}

In this chapter we define cosimplicial objects, the totalization of a cosimplicial space, and provide some examples. We assume some familiarity with simplicial objects, for they are ubiquitous throughout algebraic topology. 

\subsection{Definitions} The \emph{simplex category}, denoted $\mathbf{\Delta}$, is the category with
\begin{enumerate}
    \item objects: finite nonempty totally ordered sets. We write $[n]$ for the set $\{0<1<\cdots < n\}$,
    \item morphisms: order-preserving maps, i.e. if $i\leq j$ then $f(i)\leq f(j)$.
\end{enumerate}  
Then a \emph{cosimplicial object} in a category $\mathcal{C}$ is a functor $X^\bullet: \mathbf{\Delta}\to \mathcal{C}$. We denote the image of $[n]$ under $X^\bullet$ by $X^n$. A morphism of two simplicial objects $X^\bullet, Y^\bullet: \mathbf{\Delta}\to \mathcal{C}$ is a natural transformation of functors, i.e. morphisms $X^n\to Y^n$ for all $n$ that commute with morphisms in $\mathbf{\Delta}$. We will denote the category of of cosimplicial objects in $\mathcal{C}$ by $\catname{cs}\mathcal{C}$. In this thesis we will concern ourselves with the cases where $\mathcal{C} = \catname{Top}$ or $\mathcal{C} = \catname{Set}$.
\begin{rem}
Even though simplicial objects are \textit{contravariant} functors, the convention is to denote the simplicies using subscripts, e.g. $X_n$. Conversely, even though simplicial spaces are \textit{covariant} functors, the convention is to denote the cosimplicies with superscripts. Maybe the presence of the suffix co- explains this.
\end{rem}
Recall that the morphisms in the simplex category are generated by two distinguished classes of maps. For $n\geq 1$ and $j\in [n]$, we have injections $\delta_j: [n-1] \to [n]$ where $\delta_j$ skips $j \in [n]$. For $n\geq 0$ and $j\in [n]$, we have $n+1$ surjections $\sigma_j: [n+1]\to [n]$ where $\sigma_j$ sends both $j, j+1$ to $j \in [n]$. For a cosimplicial object $X^\bullet: \mathbf{\Delta} \to\mathcal{C}$, we call the images of the $\delta_j$ \textit{coface maps}, usually denoted $d^j$, and the images of the $\sigma_j$ \textit{codegeneracy maps}, usually denoted $s^j$. So to specify a cosimplicial object in $\mathcal{C}$ it also suffices to list a sequence of objects $X^n \in \mathcal{C}$ for $n\geq 0$, as well as coface and codegeneracy maps satisfying the following \emph{cosimplicial identities}:
\begin{enumerate}
      \item If $i<j$, then $d^j \circ d^i = d^i \circ d^{j-1}$.
      \item If $i<j$, then $s^j \circ d^i = d^i \circ s_{j-1}$.
      \item $\id = s^j \circ d^j = s^j \circ d^{j+1}$.
      \item If $i > j+1$, then $s^j \circ d^i = d^{i-1}\circ s^j$.
      \item If $i\leq j$, then $s^j \circ s^i = s^i \circ s^{j+1}$.
\end{enumerate}
One should think of a cosimplicial object $X^\bullet: \mathbf{\Delta}\to \mathcal{C}$ as a diagram 
\[\begin{tikzcd}[ampersand replacement=\&]
X^0\arrow[r, COSIMPLICIALaltstackar=3] \& 
X^1\arrow[r, COSIMPLICIALaltstackar=5] \&
X^2\arrow[r, COSIMPLICIALaltstackar=7] \&\cdots
\end{tikzcd}\]
where the rightward pointing arrows are the coface maps and the leftward pointing arrows are the codegeneracy maps. 

Our first example of a cosimplicial space will be the topological simplicies, $\Delta^\bullet: \mathbf{\Delta} \to \catname{Top}$. For each $n$, let $\Delta^n$ be the topological $n$-simplex
\[\Delta^n:= \{(x_1, ..., x_n) \in \R^{n}:  0\leq x_1 \leq \cdots \leq x_n \leq 1\}.\]
Then the coface maps $\Delta^{n-1}\to \Delta^n$ are the inclusions of faces, where $d^j$ is the inclusion of the face opposite the $j$th vertex. The codegeneracy map $s^j$ collapses the line joining the $j$th and $j$th vertex. In coordinates, we have
\[d^j(x_1, ..., x_n) = (x_1, ..., x_j, x_j, ..., x_{n-1}),\]
\[s^j(x_1, ..., x_{n}) = (x_1, ..., \widehat{x_j}, ..., x_n)\]
where $\widehat{\phantom{-}}$ indicates omission. This example is illustrative; in general it helps to think of the coface maps as ``duplicating a coordinate'' and the codegeneracy maps as ``forgetting a coordinate.'' We will see this again in this following subchapter.

\subsection{Totalization and path spaces} 
Cosimplicial spaces provide a useful model for many types of topological spaces, including the based path and loop spaces. This is done via \emph{totalization}, which is dual to the notion of geometric realization of a simplicial set. 

Given a cosimplicial space $X^\bullet: \mathbf{\Delta}\to \catname{Top}$, define the \emph{totalization} of $X^\bullet$ to be the space of maps from the cosimplicial simplices to $X^\bullet$:
\[\tot(X^\bullet) := \Hom_{\catname{csTop}}(\Delta^\bullet, X^\bullet),\]
i.e. maps $f^n: \Delta^n \to X^n$ for all $n\geq 0$ that commute with the coface and codegeneracy maps. We topologize it as a subspace of $\prod_{n\geq 0} \Hom(\Delta^n, X^n)$ with the compact-open topology. Thus totalization gives us a functor from $\catname{csTop}$ to $\catname{Top}$. 

Now for an important example which will feature later. Let $X$ be a topological space with $a,b \in X$. Define the cosimplicial space $F_{a,b}^\bullet X$ whose cosimplicies are 
\[F_{a,b}^0 X = \{*\}, \quad F_{a,b}^nX = X^n \text{ for $n\geq 1$.}\]
The coface maps $d^j: F_{a,b}^{n-1}X\to F_{a,b}^{n}$ are given by
\[d^j(x_1, ..., x_{n-1}) = \begin{cases}
  (a, x_1, ..., x_{n-1}) & j=0\\
  (x_1, ..., x_j, x_j, ..., x_{n-1}) & j \in \{1,..., n-1\}\\
  (x_1, ..., x_{n-1}, b) & j=n
\end{cases}\]
The codegeneracy maps $s^j: F_{a,b}^{n+1}X \to F_{a,b}^{n}X$ are given by
\[s^j(x_1, ..., x_{n+1}) =(x_1, ...,\widehat{x_{j+1}}, ... x_{n+1}), \quad j \in \{0, ..., n\}\]
where the $\widehat{\phantom{-}}$ denotes omission. What seems to be happening here is that by iterating the face maps, we are creating finer and finer piecewise subdivisons of paths whose endpoints are at $a$ and $b$. Indeed, 
\begin{prop}
  $\tot(P_{a,b}^\bullet X)$ is homeomorphic to the path space $\Omega_{a,b}X$ of paths in $X$ beginning at $a$ and ending at $b$.
  \begin{proof}
    A point of $\tot(P_{a,b}^\bullet X) = \hom_{\catname{csTop}}(\Delta^\bullet, P_{a,b}^\bullet X)$ us a sequence of continuous maps 
    \[f= \{f_i: \Delta^i \to X^i\}_{i \geq 0}\]
    commuting with the coface and codegeneracy maps. Fix $n\geq 2$ and $k \in \{1, ..., n\}$. Consider the following composition of codegeneracy maps
    \[\alpha_{n,k} =  \underbrace{s^{n-1} \circ s^{n-2} \circ \cdots \circ s^{k-2} \circ s^k\circ \cdots \circ s^0}_{\text{$n-1$ maps}}\]
    where we compose all the degeneracies except for $s^{k-1}$. This gives us a map $\Delta^n\to \Delta^1$ and likewise for $X^n\to X^1$. Then for $f  = \{f_0, f_1, ...\} \in \tot(P_{a,b}^\bullet X)$, we have by commutativity that 
    \[f_1 \circ \alpha_{n,k}  = \alpha_{n+1,k}\circ f_n.\]
    But now the right hand side is just picking out the $k$th coordinate of $f_n$. So we have shown that $f_n$ for $n\geq 2$ is completely determined by $f_1$, so that the projection $\Phi: \tot(P_{a,b}^\bullet X)\to \Map(\Delta^1, X)$ given by $\{f_i\}_{i\geq 0} \mapsto f_1$ is injective.

    We claim next that $\Phi$ is actually a map into $\Omega_{a,b}X\subset \Map(\Delta^1, X)$. To see this, the cosimplicial relations imply
    \[f_1\circ d^0 = d^0 \circ f_0 = \const_a, \quad f_1 \circ d^1 = d^1\circ f_0 = \const_b\]
    so $f_1(0) = a$ and $f_1(1) = b$ as desired. 
    
    Lastly we define an inverse to $\Phi: \tot(P^\bullet_{a,b}X) \to \Omega_{a,b}X$. For a given path $\gamma \in \Omega_{a,b}X$ consider the family of maps $\{f_i: \Delta^i \to X^i\}_{i\geq 0}$ given as follows. We let $f^0$ be the constant map, $f^1 = \gamma$, and for $n \geq 2$ define
    \[f_n(x_1, ..., x_n) = (\gamma(x_1), \gamma(x_1+x_2), ..., \gamma(x_1 + \cdots + x_n)).\]
    Clearly this is an inverse to $\Phi$. We leave it to the reader to check that the family of maps $\{f_i\}$ commutes with the coface and codegeneracies, and that $\Phi$ and its inverse are continuous maps. 
  \end{proof}
\end{prop}

We will call $P_{a,b}^\bullet X$ a cosimplicial model for the path space of $X$. When $a=b$ we get a cosimplicial model for the based loop space of $X$.



\subsection{Dold--Kan and homotopy} ``... there is no interesting homotopy theory of cosimplicial sets!'' concludes a  MathOverflow answer \cite{Goodwillie} of Tom Goodwillie. This subchapter will be devoted to understanding this comment.