\section{Cosimplicial objects}
\epigraph{\textit{A cosimplicial object of a category $\mathcal{C}$ could be defined simply as a simplicial object of the opposite category $\mathcal{C}^\mathrm{op}$. This is not really how the human brain works...}}{---Stacks Project, 14.5 \cite{stacks-project}}

In this chapter we define cosimplicial objects, the totalization of a cosimplicial space, and some examples. We assume some familiarity with simplicial objects, for they are ubiquitous throughout algebraic topology. 

\subsection{Definitions} The \emph{simplex category}, denoted $\mathbf{\Delta}$, is the category with
\begin{enumerate}
    \item objects: finite nonempty totally ordered sets. We write $[n]$ for the set $\{0<1<\cdots < n\}$,
    \item morphisms: order-preserving maps, i.e. if $i\leq j$ then $f(i)\leq f(j)$.
\end{enumerate}  
Then a \emph{cosimplicial object} in a category $\mathcal{C}$ is a functor $X^\bullet: \mathbf{\Delta}\to \mathcal{C}$. We denote the image of $[n]$ under $X^\bullet$ by $X^n$. A morphism of two simplicial objects $X^\bullet, Y^\bullet: \mathbf{\Delta}\to \mathcal{C}$ is a natural transformation of functors, i.e. morphisms $X^n\to Y^n$ for all $n$ that commute with morphisms in $\mathbf{\Delta}$. We will denote the category of of cosimplicial objects in $\mathcal{C}$ by $\catname{cs}\mathcal{C}$. In this thesis we will concern ourselves with the cases where $\mathcal{C} = \catname{Top}$ or $\mathcal{C} = \catname{Set}$.
\begin{rem}
Even though simplicial objects are \textit{contravariant} functors, the convention is to denote the simplicies using subscripts, e.g. $X_n$. Conversely, even though simplicial spaces are \textit{covariant} functors, the convention is to denote the cosimplicies with superscripts. Maybe the presence of the suffix co- explains this.
\end{rem}
Recall that the morphisms in the simplex category are generated by two distinguished classes of maps. For $n\geq 1$ and $j\in [n]$, we have injections $\delta_j: [n-1] \to [n]$ where $\delta_j$ skips $j \in [n]$. For $n\geq 0$ and $j\in [n]$, we have $n+1$ surjections $\sigma_j: [n+1]\to [n]$ where $\sigma_j$ sends both $j, j+1$ to $j \in [n]$. For a cosimplicial object $X^\bullet: \mathbf{\Delta} \to\mathcal{C}$, we call the images of the $\delta_j$ \textit{coface maps}, usually denoted $d^j$, and the images of the $\sigma_j$ \textit{codegeneracy maps}, usually denoted $s^j$. So to specify a cosimplicial object in $\mathcal{C}$ it also suffices to list a sequence of objects $X^n \in \mathcal{C}$ for $n\geq 0$, as well as coface and codegeneracy maps satisfying the following \emph{cosimplicial identities}:
\begin{enumerate}
      \item If $i<j$, then $d^j \circ d^i = d^i \circ d^{j-1}$.
      \item If $i<j$, then $s^j \circ d^i = d^i \circ s_{j-1}$.
      \item $\id = s^j \circ d^j = s^j \circ d^{j+1}$.
      \item If $i > j+1$, then $s^j \circ d^i = d^{i-1}\circ s^j$.
      \item If $i\leq j$, then $s^j \circ s^i = s^i \circ s^{j+1}$.
\end{enumerate}
One should think of a cosimplicial object $X^\bullet: \mathbf{\Delta}\to \mathcal{C}$ as a diagram 
\[\begin{tikzcd}[ampersand replacement=\&]
X^0\arrow[r, COSIMPLICIALaltstackar=3] \& 
X^1\arrow[r, COSIMPLICIALaltstackar=5] \&
X^2\arrow[r, COSIMPLICIALaltstackar=7] \&\cdots
\end{tikzcd}\]
where the rightward pointing arrows are the coface maps and the leftward pointing arrows are the codegeneracy maps. As will become clear in the following examples, we should think of the coface maps as ``inserting a basepoint'' and the codegeneracy maps as ``forgetting a coordinate.''

Our first example of a cosimplicial space will be the topological simplicies, $\Delta^\bullet: \mathbf{\Delta} \to \catname{Top}$. For each $n$, let $\Delta^n$ be the topological $n$-simplex
\[\Delta^n:= \{(x_0, ..., x_n) \in \R^{n+1} | \sum_{i=0}^n x_i = 1, x_i \geq 0\}.\]
Then the coface maps $\Delta^{n-1}\to \Delta^n$ are the inclusions of faces, where $d^j$ is the inclusion of the face opposite the $j$th vertex. The codegeneracy map $s^j$ collapses the line joining the $j$th and $j$th vertex. In coordinates, we have
\[d^j(x_0, ..., x_{n-1}) = (x_0, ..., x_{j-1}, 0, x_j, ..., x_{n-1}),\]
\[s^j(x_0, ..., x_{n}) = (x_0, ..., x_j +x_{j+1}, ..., x_n).\]

\subsection{Totalization and path spaces} 
Cosimplicial spaces provide a useful model for based path and loop spaces. The process whereby this is done is \emph{totalization}, which is dual to the geometric realization of a simplicial set. 

Given a cosimplicial space $X^\bullet: \mathbf{\Delta}\to \catname{Top}$, define the \emph{totalization} of $X^\bullet$ to be the space of maps from the cosimplicial simplices to $X^\bullet$:
\[\tot(X^\bullet) := \Hom_{\catname{csTop}}(\Delta^\bullet, X^\bullet),\]
i.e. maps $f^n: \Delta^n \to X^n$ for all $n\geq 0$ that commute with the coface and codegeneracy maps. We topologize it as a subspace of $\prod_{n\geq 0} \Hom(\Delta^n, X^n)$ with the compact-open topology. Thus totalization gives us a functor from $\catname{csTop}$ to $\catname{Top}$. 

Now for an important example which will feature in later proof. Let $X$ be a topological space with $a,b \in X$. Define the cosimplicial space $F_{a,b}^\bullet X$ whose cosimplicies are 
\[F_{a,b}^0 X = \{*\}, \quad F_{a,b}^nX = X^n \text{ for $n\geq 1$.}\]
The coface maps $d^j: F_{a,b}^{n-1}X\to F_{a,b}^{n}$ are given by
\[d^j(x_1, ..., x_{n-1}) = \begin{cases}
  (a, x_1, ..., x_{n-1}) & j=0\\
  (x_1, ..., x_j, x_j, ..., x_{n-1}) & j \in \{1,..., n-1\}\\
  (x_1, ..., x_{n-1}, b) & j=n
\end{cases}\]
The codegeneracy maps $s^j: F_{a,b}^{n+1}X \to F_{a,b}^{n}X$ are given by
\[s^j(x_1, ..., x_{n+1}) =(x_1, ...,\widehat{x_{j+1}}, ... x_{n+1}), \quad j \in \{0, ..., n\}\]
where the $\widehat{\phantom{-}}$ denotes omission. Before we compute the totalization let us try to give some intuition.


\subsection{Dold--Kan and homotopy} ``... there is no interesting homotopy theory of cosimplicial sets!'' concludes a  MathOverflow answer \cite{Goodwillie} of Tom Goodwillie. This subchapter will be devoted to understanding this comment.