\section{(Co)algebras}
In this chapter we 
\subsection{Algebras} In a first course in ring theory, one encounters, given a base ring $R$, the ring of polynomials $R[x]$. They come with an addition and multiplication determined by the ring operations of $R$. In other words, $R[x]$ is an \emph{algebra} over $R$. In general, an \emph{associative $R$-algebra} is a ring $A$ that is also an $R$-module, such that the ring product is $R$-bilinear. That is, for all $r \in R$ and $x,y \in A$:
\[r \cdot (xy) = (r\cdot x)y = x(r\cdot y).\]
\begin{rem}
    In most literature, the ring product $A\times A \to A$ of an $R$-algebra $A$ is instead written as a map $A\tensor_{R} A\to A$. By the universal property of tensor products these presentations are equivalent.
\end{rem}
A \emph{morphism} of associative $R$-algebras $f: A\to B$ is a map respecting both the $R$-module structure and ring structure, i.e. for $r\in R$ and $a, b \in A$, we require 
\[r\cdot f(ab) = f(r\cdot ab) = r\cdot f(a)f(b).\]
 An algebra morphism from $R$-algebra $A$ to $R$ itself is called an \textit{augmentation}. In this case we say $A$ is \emph{augmented}.

\subsubsection{Group rings} One important class of algebras are group rings. For a fixed ring $R$ and a group $G$, the \emph{group ring of $G$ over $R$}, denoted $R[G]$, has elements finite formal linear combinations of elements in $G$ with coefficients in $R$, with addition and multiplication given by
\[(\sum_{g \in G} r_g g)+ (\sum_{g \in G}s_g g) = \sum_{g \in G}(r_g+s_g)g,\quad (\sum_{g \in G} r_g g)(\sum_{g \in G}s_g g) = \sum_{g \in G}\sum_{g_1g_2=g} (r_{g_1}s_{g_2})g.\]
Then the action of $R$ on $R[G]$ given by multipliying coefficients gives $R[G]$ the structure of an $R$-algebra. One should think of $R[G]$ as some sort of free module over $R$ with basis $G$. Group rings abound in the representation theory of groups, where any representation $\rho: G\to \GL(V)$ of a group $G$ over a $k$-vector space $V$ corresponds to a module over the group ring $k[G]$. In this thesis, however, we do not need to take this perspective (unless we could?)

As an algebra, $R[G]$ comes with a natural augmentation, given by the map
\[\varepsilon: R[G]\to R,\quad \varepsilon(\sum_{g\in G}r_gg) = \sum_{g\in G}r_g.\]
We call its kernel the \emph{augmentation ideal}, and it has generating set 
\[\{g-1: g\in G\}\]
where $g-1$ is the group ring element $1_Rg - 1_R1_G$. The augmentation ideal is an interesting object of study. For example, we have the following observation: 
\begin{prop}
    Let $\mathcal{I}$ be the augmentation ideal of the integral group ring $\Z[G]$. Then $\mathcal{I}/\mathcal{I}^2\cong G^\mathrm{ab}$, the abelianzation of $G$.
    \begin{proof}
        The proof relies on the following two facts. First, that $\{g-1: g\in G\}$ is a generating set of $\mathcal{I}$. Second, that the abelianzation of $G$ is the quotient of $G$ by its commutator subgroup $[G,G]$, which is generated by group elements of the form $g^{-1}h^{-1}gh$. Then we can define an explicit homomorphism
        \[\mathcal{I}/\mathcal{I}^2 \to G^\mathrm{ab} = G/[G,G], \quad [g-1] \mapsto [g]\]
        and extending linearly to all of $\mathcal{I}/\mathcal{I}^2$. Then the inverse map is $[g] \to [g-1]$.
    \end{proof}
\end{prop}
Given a path connected, based topological space with $(X,x)$, consider its integral fundamental group ring $\Z\pi_1(X,x)$ and corresponding augmentation ideal $\mathcal{I}$. The previous proposition implies that 
\[\mathcal{I}/\mathcal{I}^2 \cong \pi_1(X,x)^\mathrm{ab} \cong H_1(X; \Z).\]
This is the simplest case of the main theorem we are trying to prove. It hints at the fact that we can gain information about the fundamental group of the space by examining its homology.

\subsubsection{Differential graded algebras} Many algebras have extra differential graded structure. Or perhaps many differential graded rings have extra algebra strucutre. For example, consider the singular chains $C_\bullet(X)$ on a topological space $X$. They are $\N$-graded, where each $C_n(X)$ is the free abelian group on the set of $n$-simplicies of $X$:
\[C_n(X) = \Z\{\text{continuous maps $\Delta^n\to X$}\}.\]
An alternative name for $\Z$-module is ``abelian group,'' which 
\[C_\bullet = \bigoplus_{n\in\N}C_n(X)\]
is. Next, we have a product 
\[C_p(X)\times C_q(X)\to C_{p+q}(X)\]
given by the decomposition of the geometric $(p+q)$-simplex (details in []) which assemble to a product 
\[C_\bullet(X)\times C_\bullet(X) \to C_\bullet(X).\]
One can check that this product is compatible with the $\Z$-module (i.e. abelian group) structure on $C_\bullet(X)$, making it into a \textit{graded $\Z$-algebra}. As icing on the cake, the singular boundary map $\partial: C_n(X)\to C_{n-1}(X)$ is a differential, i.e. $\partial^2=0$, and 

\subsection{Coalgebras} Occuring less in nature are coalgebras. 

free 
tensor coalgebra
cdga