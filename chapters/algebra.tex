\section{(Co)algebras and the cobar construction}
In this chapter we give a brief overview of the important algebraic constructions used throughout the thesis. The reader should feel free to skip this the first two subsections if they are familiar with the theory of (co)associative (co)algebras. We then state Adams' remarkable 1956 theorem \cite{AdamsCobar}, which (roughly) says that given a based topological space $(X,x)$, one can recover the singular chains of the based loop space $\Omega_xX$ via a purely algebraic construction on the singular chains on $X$. We conclude with some recent refinements of Adams' theorem. The primary reference for the background material is \cite{Loday2012}.

\subsection{Algebras} In a first course in ring theory, one encounters, given a base ring $R$, the ring of polynomials $R[x]$. They come with an addition and multiplication determined by the ring operations of $R$. In other words, $R[x]$ is an \emph{algebra} over $R$. In general, an \emph{associative $R$-algebra} is a (possibly non-unital) ring $A$ that is also an $R$-module, such that the ring product is $R$-bilinear. That is, for all $r \in R$ and $x,y \in A$:
\[r \cdot (xy) = (r\cdot x)y = x(r\cdot y).\]
\begin{rem}
    In much literature, the ring product $A\times A \to A$ of an $R$-algebra $A$ is instead written as a map $A\tensor_{R} A\to A$. By the universal property of tensor products these presentations are equivalent.
\end{rem}
If $A$ is unital, then the algebra $A$ is called \emph{unital}. A \emph{morphism} of associative $R$-algebras $f: A\to B$ is a map respecting both the $R$-module structure and ring structure, i.e. for $r\in R$ and $a, b \in A$, we require 
\[r\cdot f(ab) = f(r\cdot ab) = r\cdot f(a)f(b).\]
 An algebra morphism from $R$-algebra $A$ to $R$ itself is called an \textit{augmentation}. In this case we say $A$ is \emph{augmented}.

\subsubsection{Group rings} One important class of algebras are group rings. For a fixed ring $R$ and a group $G$, the \emph{group ring of $G$ over $R$}, denoted $R[G]$, has elements finite formal linear combinations of elements in $G$ with coefficients in $R$, with addition and multiplication given by
\[(\sum_{g \in G} r_g g)+ (\sum_{g \in G}s_g g) = \sum_{g \in G}(r_g+s_g)g,\quad (\sum_{g \in G} r_g g)(\sum_{g \in G}s_g g) = \sum_{g \in G}\sum_{g_1g_2=g} (r_{g_1}s_{g_2})g.\]
Then the action of $R$ on $R[G]$ given by multipliying coefficients gives $R[G]$ the structure of an $R$-algebra. One should think of $R[G]$ as some sort of free module over $R$ with basis $G$. Group rings abound in the representation theory of groups, where any representation $\rho: G\to \GL(V)$ of a group $G$ over a $k$-vector space $V$ corresponds to a module over the group ring $k[G]$. In this thesis, however, we do not need to take this perspective (unless we could?)

As an algebra, $R[G]$ comes with a natural augmentation, given by the map
\[\varepsilon: R[G]\to R,\quad \varepsilon(\sum_{g\in G}r_gg) = \sum_{g\in G}r_g.\]
We call its kernel the \emph{augmentation ideal}, and it has generating set 
\[\{g-1: g\in G\}\]
where $g-1$ is the group ring element $1_Rg - 1_R1_G$. The augmentation ideal is an interesting object of study. For example, we have the following observation: 
\begin{prop}
    Let $\mathcal{I}$ be the augmentation ideal of the integral group ring $\Z[G]$. Then $\mathcal{I}/\mathcal{I}^2\cong G^\mathrm{ab}$, the abelianzation of $G$.
    \begin{proof}
        The proof relies on the following two facts. First, that $\{g-1: g\in G\}$ is a generating set of $\mathcal{I}$. Second, that the abelianzation of $G$ is the quotient of $G$ by its commutator subgroup $[G,G]$, which is generated by group elements of the form $g^{-1}h^{-1}gh$. Then we can define an explicit homomorphism
        \[\mathcal{I}/\mathcal{I}^2 \to G^\mathrm{ab} = G/[G,G], \quad [g-1] \mapsto [g]\]
        and extending linearly to all of $\mathcal{I}/\mathcal{I}^2$. Then the inverse map is $[g] \to [g-1]$.
    \end{proof}
\end{prop}
Given a path connected, based topological space with $(X,x)$, consider its integral fundamental group ring $\Z\pi_1(X,x)$ and corresponding augmentation ideal $\mathcal{I}$. The previous proposition implies that 
\[\mathcal{I}/\mathcal{I}^2 \cong \pi_1(X,x)^\mathrm{ab} \cong H_1(X; \Z).\]
This is the simplest case of the main theorem we are trying to prove. It hints at the fact that we can gain information about the fundamental group of the space by examining its homology.

\subsubsection{Differential graded algebras} We begin with an example. Consider the singular chains $C_\bullet(X)$ on a topological space $X$. They are $\N$-graded, where each $C_n(X)$ is the free abelian group on the set of $n$-simplicies of $X$:
\[C_n(X) = \Z\{\text{continuous maps $\Delta^n\to X$}\}.\]
Another name given to $\Z$-modules is ``abelian group,'' and indeed 
\[C_\bullet = \bigoplus_{n\in\N}C_n(X)\]
is an abelian group. Next, we have a product 
\[C_p(X)\times C_q(X)\to C_{p+q}(X)\]
given by the decomposition of the geometric $(p+q)$-simplex into a sum of $p$-simplices and $q$-simplicies (see []) for a reference. These maps assemble to a product 
\[\times: C_\bullet(X)\times C_\bullet(X) \to C_\bullet(X).\]
One can check that this product is compatible with the $\Z$-module structure on $C_\bullet(X)$, making it into a \emph{graded $\Z$-algebra}. As icing on the cake, we also have a singular boundary map $\partial: C_n(X)\to C_{n-1}(X)$ satisfying $\partial^2=0$ and the graded Leibniz rule: for $\sigma \in C_p(X)$ and $\tau \in C_q(X)$, 
\[\partial(\sigma\times \tau)= (\partial\sigma)\times \tau + (-1)^p \sigma \times (\partial\tau).\]
Thus $C_\bullet(X)$ has the structure of a \emph{differential graded $\Z$-algebra}. In general, a \emph{differential graded $R$-algebra}, often abbreviated as a \emph{dg algebra}, is a graded $R$-algebra $A_\bullet$ with a differential satisfying the graded Leibniz rule. If the differential lowers degree, we say $A_\bullet$ is \emph{homologically graded}; if it raises degree, we say $A_\bullet$ is \emph{cohomologically graded}. We can think of dg algebras as (co)chain complexes with a product, or algebras with a chain structure. 

\subsubsection{Freeness and the tensor algebra}
Just as we can construct a free group from a set, we can also construct a free $R$-algebra given any $R$-module $A$. Before we given an explicit construction, we give a description of the \emph{free associative algebra over $A$}, denoted $\mathcal{F}A$, in terms of its \emph{universal property}:
\begin{quote}
    \textit{There is an $R$-linear map $i: A\hookrightarrow \mathcal{F}A$ such that any $R$-algebra morphism $f: A\to B$ extends into a unique morphism $\tilde{f}: \mathcal{F}A\to B$ with $\tilde{f}\circ i = f$.}
\end{quote}
This is entirely analogous to universal property of a free group. Sending an $R$-module $A$ to the free associative algebra over $A$ gives us a functor from the category of $R$-modules (supposing $R$ is commutative), $\catname{Mod}_R$ to the category of associative $R$-algebras, denoted $\catname{Alg}_R$. Conversely, we have a forgetful functor sending a $R$-algebra to its underlying module. In this language, the universal property tells us that these two functors are \emph{adjoint}, i.e. there is a natural isomorphism
\[\hom_{\catname{Alg}_R}(\mathcal{F}A, B) \cong \hom_{\catname{Mod}_R}(A, \catname{forget}(B)).\]
This universal property guarantees that two manifestations of the free associative algebra are isomorphic via a unique isomorphism. With this in mind, let us now define the \emph{tensor algebra} over an $R$-module $A$. Denoted $T(A)$, its underlying $R$-module is given by
\[T(A) := R\oplus A \oplus A^{\tensor 2} \oplus \cdots\]
and the product $T(A)\tensor T(A)\to T(A)$ is given by concatenation. On homogenous tensors $a \in A^{\tensor p}$ and $b \in A^{\tensor q}$, we have 
\[(a_1\cdots a_p)\tensor (b_1 \cdots b_q) \mapsto (a_1 \cdots a_p b_1 \cdots b_q) \in A^{\tensor p+q}.\]
This is a unital and associative $R$-algebra, with augmentation given by 
\begin{prop}
    $T(A)$ is the free associative algebra over $A$.
    \begin{proof}
        
    \end{proof}
\end{prop}

\subsection{Coalgebras} Occuring less in nature are coalgebras. 

free 
tensor coalgebra
cdga